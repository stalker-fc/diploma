\newpage
\section*{\LARGE{Выводы}}
\addcontentsline{toc}{section}{Выводы}
\hskip 12 mm
В ходе данной работы была составлена математическая модель задачи проектирования линейных объектов. Общая задача была разбита на два этапа: решение задачи оптимального построения линейного объекта и решение задачи построения оптимальных сетей.
\par
В рамках первого этапа задача была сведена к поиску кратчайшего пути на графе. Было сгенерировано несколько модельных карт местности, для которых было найдено аналитическое решение и проведено сравнение с результатами полученными численно. Наименьшее значение ошибки для всех модельных карт составило менеее 0.1\% за время не превышающее 85 секунд.
\par
В рамках второго этапа был реализован свой алгоритм на основе знаний о предметной области. Была найдена зависимость отношения ошибки между численным и аналитическим решением, временем работы алгоритма, количеством точек разветвления. На основании чего был сделан вывод о том, что можно получать решения с величиной ошибки около 1\% с количеством точек разветвления менее 0.1\% от общего числа точек в графе за время менее 130 секунд.
\newpage
\section*{\LARGE\bf{Заключение}}
\addcontentsline{toc}{section}{Заключение}
\hskip 12 mm
В рамках данной работы была исследована проблема поиска оптимального пути между двумя объектами, был предложен и реализован алгоритм, позволяющий строить оптимальные сети линейных объектов с учетом рельефа и различных типов местности. Также были предложены различные случаи для тестирования корректности работы программы.
\par
В качестве дальнейшего направления работы можно выделить повышение точности построенного маршрута, уменьшения времени работы алгоритма, введение дополнительных параметров, используемых в расчёте.
