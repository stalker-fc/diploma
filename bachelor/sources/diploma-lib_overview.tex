\newpage
\section*{\LARGE{Обзор литературы}}
\addcontentsline{toc}{section}{Обзор литературы}
\hskip 12 mm
Проблема соединения нескольких точек лежащих на плоскости системой дорог наименьшей суммарной длины таким образом, что из каждой точки можно было добраться в любую другую возникла еще очень давно. Её постановка для случая $n = 3$ была предложена математиком П. Ферма, а первое решение было получено Б. Кавальери и Э. Торричелли. Данную задачу можно считать одной из старейших оптимизационных проблем в математике. В 1934 году В. Ярник и О. Кесслер обобщили эту задачу для произвольных $n$ точек. А в 1941 году вышла книга "Что такое математика?"\, в которой авторы называют эти две задачи проблемой Штейнера в честь математика Я. Штейнера, первым получившим чисто геометрическое решение задачи Ферма. Книга была достаточно популярной, поэтому название задачи закрепилось за именем Штейнера \cite{SteinerProblemWiki}.
\par
Если в ранние годы задача формулировалась строго математически, то двадцатый век принес её в прикладную область. В настоящее время можно выделить несколько различных постановок задачи Штейнера, для каждой из которых используются свои методы решения. В классической постановке задачи используется понятие Евклидовой плоскости, а, соответственно, и Евклидовой метрики. Она может быть интересна в теоретическом плане. Вторая постановка задачи предполагает использование прямоугольной метрики, что позволяет снизить затраты при проектировании сверх-больших интегральных схем. Третья постановка задачи базируется на теории графов, и она является наиболее широкой в применении. С помощью решения задачи в этой постановке можно повысить эффективность проектирования коммуникационных, электрических и механических сетей. Задача Штейнера является NP-полной, и к настоящему моменту не существует алгоритмов для любой из постановок задачи, способных решить её за полиномиальное время \cite{SteinerOverview}.
\par
Решение данной задачи до сих пор является актуальным, что подтверждают работы, как российских, так и зарубежных ученых. Например, для неориентированных графов данная задача рассматривалась Хакими в 1971 году, Дрейфусом и Вагнером в 1972 году и Такахаши и Матцуямой в 1980 году. Данные алгоритмы являются вычислительно неэффективными, хотя они и позволяют находить точное решение \cite{SteinerProblemInGraphs}. В 1980 году  Д.Т. Лотарев сталкивался с задачей проектирования транспортной сети, но в своей работе он рассматривал задачу с учетом потоков в графе  \cite{Lotarev1980}. 
\par
Также помимо публикаций отдельных научных работ в 2014 году проходила конференция-соревновнование DIMACS-11, на которой участники соревновались в скорости работы их алгоритмов для решения задачи Штейнера в различных её постановках \cite{Dimacs11}.
