\newpage
\section*{\LARGE{Постановка задачи}}
\addcontentsline{toc}{section}{Постановка задачи}
\hskip 12 mm
Задача построения оптимальных трасс линейных объектов выглядит следующим образом. Пусть задан определённый участок поверхности Земли вместе с соответствующим ему рельефом. Также на этом участке присутствуют области различных типов местности, такие как леса, болота, реки и т.п., с заданной для каждого типа местности стоимостью строительства и имеющий вид многоугольников. Помимо этого присутствуют уже существующие линейные объекты, такие как дороги, трубопроводы и т.п., которые можно представить в виде ломаных линий с заданной стоимостью строительства на них. И также задан набор объектов, представляющих собой точки земной поверхности, которые нужно связать оптимальной сетью линейных объектов.
\par
В математическом виде задачу можно представить так: задана поверхность в трёхмерном пространстве $\hat{S}$, описывающая земной рельеф, также задано множество объектов $\hat{P} \in \hat{S}$. Требуется связать множество объектов $\hat{P}$ коммуникационной сетью так, чтобы затраты на строительство данной сети были минимальны.
\par
Обозначим за область расчетов $Q$ проекцию поверхности $\hat{S}$ на плоскость $xOy$. 
При проектировании  $\hat{P}$ на $Q$, будет получено множество координат точек $P = \{(x_i, y_i) \in Q , i=\overline{1..n}\}$. Помимо области проектирования и объектов также заданы множество областей $A = \{\{x_{j},y_{j}\}_{j=1}^{n_i} \in Q, i=\overline{1..m}\}$ в виде произвольных многоугольников и множество существующих линейных объектов \mbox{$L = \{\{x_{j},y_{j}\}_{j=1}^{n_i} \in Q, i=\overline{1..k}\}$} в виде ломаных линий. 
Основываясь на стоимостях строительства в областях $A$ и вдоль линейных объектов $L$ задана функция строительных затрат $c: \{Q \rightarrow \mathbb{R}, \forall(x, y) \in Q: c(x, y) \ge 0\}$, обозначающая стоимость строительства в каждой конкретной точке плоскости $Q$, а также функция задающая высоту точки $h: \{Q \rightarrow \mathbb{R}, \forall(x, y) \in Q\}$.
\par
Целью данной работы является написание программы в результате работы которой должна быть построена сеть линейных объектов, представляющая собой набор ломаных линий $N = \{\{x_{j},y_{j}\}_{j=1}^{n_i} \in Q, i=\overline{1..s}\}$ оптимальная по заданным параметрам.
\par
Для решения необходимо выполнить следующее:
\begin{enumerate}
	\item Определить способ вычисления стоимости строительства линейного объекта.
	\item Проанализировать основные алгоритмы построения пути между двумя объектами и реализовать оптимальный.
	\item Проанализировать основные алгоритмы построения сети линейных объектов и реализовать оптимальный.
	\item Исследовать различные случаи для тестирования работы программы.
\end{enumerate}

\subsection*{\Large{Актуальность проблемы}}
\hskip 12 mm
При проектировании линейного объекта инженер сталкивается с проблемой правильности составления стоимостной оценки проекта. Область проектирования представляет собой неоднородную территорию, на которой удельные затраты на строительство линейного объекта в различных ее точках различны. Значения удельных строительных затрат в значительной мере определяются типом грунтов различных категорий: болото, суходол, лес, вечномерзлые грунты, речная пойма и т.д. Помимо разного вида грунтов нужно также учитывать уже существующую инфраструктуру, ранее созданную человеком. Например, прокладка автодороги между двумя объектами с учетом уже существующей дороги, зачастую, окажется дешевле, чем если не учитывать построенную дорогу.
\par
Но помимо удешевления затрат от использования инфраструктуры существует ряд ограничений, которые могут существенно увеличить стоимость строительства. В качестве примера рассмотрим проектирование высоковольтной линии. Для каждой из них предусмотрена санитарно-защитная зона, определяющая минимальное расстояние до ближайших жилых, производственных и непроизводственных зданий и сооружений. И для различных классов линейных объектов есть свои строительные нормы и правила (СНиП), которые нужно учитывать в обязательном порядке. В том числе в них прописаны допустимые перепады высот на линейном объекте, что заставляет инженера использовать данные о рельефе местности.
\par
Если список ограничений на строительство одного линейного объекта вполне поддается аналитике, то при проектировании сети линейных объектов список факторов, влияющих на размещение точек разветвления, то есть оптимального расположения перекрёстков, становится достаточно большим, и однозначно обосновать правильность места расположения точки разветвления является нетривиальной задачей. В качестве примера рассмотрим задачу проектирования сети трубопроводов для месторождения, когда у каждой скважины есть график выработки и график выхода на проектную мощность. Стоимость строительства трубопровода зависит от диаметра трубы, а также от поддерживающей инфраструктуры месторождения, как электрические подстанции, объема использеумой воды. И если не учитывать стоимость эксплуатации при проектировании месторождения, то расчитанные местоположения точек разветвления могут показать себя крайне неэффективными \cite{GasAndOil}.
\par
Исходя из всего этого, проблема является актуальной, так как рынку требуется, как просто инструмент для аналитики стоимости строительства уже спроектированных сетей линейных объектов, так и создание системы, которая сможет строить сети линейных объектов в автоматическом режиме, учитывая ряд факторов, как эксплуатационные расходы, расходы на транспортировку строительных материалов, а также различные СНиП и требования законодательства.


