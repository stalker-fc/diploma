\section*{\Large{Определения, обозначения и сокращения}}
\addcontentsline{toc}{section}{Определения, обозначения и сокращения}

В настоящем отчете применяют следующие термины с соответствующими определениями.

\textit{Генеральный план (генплан, ГП)} в общем смысле —
проектный документ, на основании которого осуществляется планировка,
застройка, реконструкция и иные виды градостроительного освоения территорий.
Основной частью генерального плана (также называемой собственно генеральным планом)
является масштабное изображение, полученное методом графического наложения чертежа
проектируемого объекта на топографический,
инженерно-топографический или фотографический план территории.
