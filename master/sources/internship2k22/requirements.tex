\section*{\Large{Требования}}
\addcontentsline{toc}{section}{Требования}

В данном пункте кратко перечислим те функциональные и нефункциональные требования, с которыми пришлось столкнуться
в рамках реализации сервиса для запуска математических методов библиотеки, отвечающей за автоматическое формирование
генеральных планов сооружений.

\subsection*{\large{Функциональные требования}}
\addcontentsline{toc}{subsection}{Функциональные требования}

В рамках функциональных требований к сервису можно выделить следующее:
\begin{enumerate}
    \item API должен предусматривать асинхронный запуск математического метода в отдельном процессе.
    \item API должен придерживаться концепции REST.
    \item Для обмена данными с клиентом требуется использовать JSON.
    \item API должен предусматривать следующие функции для взаимодействия с расчетными задачами:
    \begin{itemize}
        \item создание задачи.
        \item запуск задачи.
        \item получение статуса выполнения задачи.
        \item отмена выполнения задачи.
        \item получение результатов выполнения задачи.
    \end{itemize}
    \item Для обмена расчетными данными следует использовать единый внутренний формат данных системы \textbf{nd\_plan\_model}

\end{enumerate}

\subsection*{\large{Нефункциональные требования}}
\addcontentsline{toc}{subsection}{Нефункциональные требования}

Из нефункциональных требований выделим следующие:
\begin{enumerate}
    \item Язык программирования \textbf{Python 3.8}
    \item Развертывание сервиса осуществлять с помощью \textbf{Docker}
\end{enumerate}
