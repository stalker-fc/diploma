\section*{\Large{Требования}}
\addcontentsline{toc}{section}{Требования}

Ниже перечислены те функциональные и нефункциональные требования
для сервиса запуска математических методов библиотеки \textbf{nd\_plan}.

\subsection*{\large{Функциональные требования}}
\addcontentsline{toc}{subsection}{Функциональные требования}

В рамках функциональных требований к сервису можно выделить следующее:
\begin{enumerate}
    \item Запуск математических методов библиотеки \textbf{nd\_plan} должен осуществляться в отдельном процессе.
    \item Запуск математических методов библиотеки \textbf{nd\_plan} должен осуществляться через API.
    \item API должен придерживаться концепции REST.
    \item API должен предусматривать асинхронный запуск математического метода.
    \item API должен использовать JSON в качестве обмена данных с клиентом.
    \item API должен уметь обрабатывать внутреннюю модель расчетных данных системы \textbf{nd\_plan\_model}.
    \item API должен предусматривать следующие функции для взаимодействия с расчетными задачами:
    \begin{itemize}
        \item создание задачи.
        \item запуск задачи.
        \item получение статуса выполнения задачи.
        \item отмена выполнения задачи.
        \item получение результатов выполнения задачи.
    \end{itemize}
\end{enumerate}

\subsection*{\large{Нефункциональные требования}}
\addcontentsline{toc}{subsection}{Нефункциональные требования}

Из нефункциональных требований выделим следующие:
\begin{enumerate}
    \item Язык программирования \textbf{Python 3.8}
    \item Развертывание сервиса осуществлять с помощью \textbf{Docker}
\end{enumerate}
