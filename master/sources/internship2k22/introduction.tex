\section*{\Large{Введение}}
\addcontentsline{toc}{section}{Введение}
Автоматическое формирование генерального плана площадного объекта капитального строительства является чрезвычайно
сложной задачей, как в алгоритмическом, так и в технологическом плане.
Генеральный план помимо расстановки сооружений содержит расположение различных трубопроводов, линий электропередач,
внутриплощадочных проездов, пожарных гидранты и прочих объектов, необходимых для функционирования того или иного
промышленного объекта.

Для того, чтобы задача была выполнена качественно, требуется применение математического аппарата. Математический
аппарат подразумевает приведение объектов, которыми оперирует бизнес, к абстракциям,
которые могут использованы в алгоритмах. Именно эти алгоритмы, которые несколько отдалены от понятий бизнеса и являются
основным ядром и функционалом системы. Все эти алгоритмы выделены в отдельную математическую библиотеку.

Как и любая математически сложная задача, она обладает следующими свойствами:
\begin{itemize}
    \item долгое время выполнения;
    \item высокая загруженность центрального процессора;
    \item высокое потребление оперативной памяти.
\end{itemize}

Требуется создать прослойку, которая сможет превращать объекты технологической модели данных в объекты математической
библиотеки, а также сможет запускать математические методы, учитывая их особенности выполнения.
