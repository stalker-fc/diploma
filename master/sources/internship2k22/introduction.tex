\section*{\Large{Введение}}
\addcontentsline{toc}{section}{Введение}
Автоматическое формирование генерального плана (ГП) площадного объекта капитального строительства является чрезвычайно
сложной задачей, как в алгоритмическом, так и в технологическом плане.
На генеральном плане помимо сооружений отражены внутриплощадочные проезды, различные трубопроводы, линии электропередач,
технологические эстакады, пожарные гидранты и прочие объекты,
необходимые для функционирования того или иного площадного объекта.

Для каждого объекта ГП заданы определенные требования к его размещению.
Так как объектов много и они очень разнообразны по своей структуре,
то использовать единый алгоритм для их расстановки невозможно.
Поэтому процесс формирования ГП состоит из последовательного набора этапов, на каждом из которых
рассчитывается местоположение определенного типа объектов.

Каждый этап представляет собой представляет собой применение одного или целого ряда алгоритмов.
Алгоритмы находятся в отдельной математической библиотеке \textbf{nd\_plan}.
Библиотека \textbf{nd\_plan} является внутренней разработкой, содержащей все алгоритмы,
применяемые для решения поставленной задачи.

Из особенностей данной библиотеки стоит отметить, что в модели данных методов используются абстрактные
структуры данных, отдаленные от предметной области понятной заказчику. Например, в рамках
математических алгоритмов, мы будем рассматривать лишь граф, имеющий определенные свойства,
а уже в терминах предметной области, данный граф станет совокупностью линий электропередач и трубопроводов.

К особенностям использования данной библиотеки стоит отнести типичные проблемы сложных алгоритмов: долгое время
выполнения, высокий уровень потребления оперативной памяти и высокую нагрузку на центральный процессор.

Для того чтобы воспользоваться математическим аппаратом требуется создать сервис, который обеспечит преобразование
данных предметной области в объекты математической библиотеки \textbf{nd\_plan},
а также будет иметь возможность запускать методы данной библиотеки, учитывая особенности их выполнения.
