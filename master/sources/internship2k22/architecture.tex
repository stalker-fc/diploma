\section*{\large{Архитектура сервиса}}
\addcontentsline{toc}{section}{Архитектура сервиса}

На диаграмме размещения системы(см. рис\ \ref{pic:architecture__deployment-diagram}) расчетный сервис выделен голубым цветом.

\begin{figure}[H]
	\hspace*{-1 cm}\includegraphics[width=\textwidth]{images/architecture/deployment_diagram}
	\caption{Диаграмма размещения}
	\label{pic:architecture__deployment-diagram}
\end{figure}
\vskip 5 mm

Расчетный сервис представлен тремя компонентами: API, расчётным модулем и математической библиотекой \textbf{nd\_plan}.
В качестве цели данной работы является проектирование и разработка API и расчетного модуля,
то поэтому только они и отражены на диаграммах ниже.

\begin{figure}[H]
	\hspace*{-2.5 cm}\includegraphics[width=1.2\textwidth]{images/architecture/execution_classes_diagram}
	\caption{Диаграмма классов расчетного модуля}
	\label{pic:architecture__execution-classes-diagram}
\end{figure}

\vskip 5 mm
Архитектура расчетного модуля представлена на диаграмме классов(см. рис\ \ref{pic:architecture__execution-classes-diagram}).
Основные классы и интерфейсы:
\begin{enumerate}
	\item \textit{Task} -- расчетная задача.
	\item \textit{TaskStatus} -- статус выполнения расчетной задачи.
	\item \textit{ITaskRepository} -- получения данных по расчётной задаче.
	\item \textit{ITaskHandler} -- запуск расчетных задач в отдельном процессе.
	\item \textit{ITaskCommandQueue} -- очередь задач.
	\item \textit{ITaskQueueListener} -- обработчик очереди задач, инициализация процесса расчета задач.
\end{enumerate}


\begin{figure}[H]
	\hspace*{-2.5 cm}\includegraphics[width=1.2\textwidth]{images/architecture/}
	\caption{Диаграмма классов API}
	\label{pic:architecture__api-classes-diagram}
\end{figure}
