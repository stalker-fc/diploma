\section*{\Large{Определения, обозначения и сокращения}}
\addcontentsline{toc}{section}{Определения, обозначения и сокращения}

В настоящем отчете применяют следующие термины с соответствующими определениями.

\textit{Генеральный план (генплан, ГП)} в общем смысле —
проектный документ, на основании которого осуществляется планировка,
застройка, реконструкция и иные виды градостроительного освоения территорий.
Основной частью генерального плана (также называемой собственно генеральным планом)
является масштабное изображение, полученное методом графического наложения чертежа
проектируемого объекта на топографический,
инженерно-топографический или фотографический план территории.

\textit{Тайловая, плиточная или знакоместная графика} — метод создания больших изображений
(как правило, уровней в компьютерных играх) из маленьких фрагментов одинаковых размеров, подобно картинам из изразцов.

\textit{Векторные тайлы} — это способ доставки географических данных небольшими порциями в
браузер или другое клиентское приложение. Векторные тайлы похожи на растровые тайлы,
но вместо растровых изображений возвращаемые данные являются векторным представлением объектов в листе.
Например, векторный тайл GeoJSON может включать дороги как LineStrings и водоёмы как Polygons.

\textit{Mapbox Vector Tile (MVT)} — формат передачи векторных тайлов.
