\section*{\Large{Реализация}}
\addcontentsline{toc}{section}{Реализация}

При реализации описанной архитектуры была получена следующая структура проекта.
\dirtree{%
.1 /.
.2 app.
.3 api.
.4 \_\_init\_\_.py.
.4 contexts.py.
.4 controller.py.
.4 model.py.
.4 responses.py.
.4 views.py.
.3 domain.
.4 \_\_init\_\_.py.
.4 cache\_storage.py.
.4 data\_provider.py.
.4 model.py.
.4 renderer.py.
.4 repository.py.
.3 execution.
.4 \_\_init\_\_.py.
.4 executor.py.
.4 handler.py.
.4 listener.py.
.4 queue.py.
.4 result.py.
.4 storage.py.
.3 repository.
.4 \_\_init\_\_.py.
.4 cache\_storage.py.
.4 data\_provider.py.
.4 renderer.py.
.4 repository.py.
.3 \_\_init\_\_.py.
.3 app.py.
.3 exceptions.py.
.3 logger.py.
.2 Dockerfile.
.2 README.md.
.2 main.py.
.2 requirements.txt.
}

Само приложение находится в директории \textbf{app}.
Всего получилось 4 python-пакета, в которых и скрыта основная логика работы.
\begin{enumerate}
    \item \textbf{api} -- реализует API тайлового сервера через HTTP. То есть endpoint-ы,
    модели запросов/ответов к серверу, а также взаимодействие с \textit{ITilesRepository}
    \item \textbf{domain} -- пакет, в котором определены только интерфейсы взаимодействия между модулями программы.
    Именно здесь и реализована диаграмма классов (см. рис\ \ref{pic:architecture__classes-diagram}). Непосредственная
    реализация обозначенных интерфейсов находится в других пакетах.
    \item \textbf{execution} -- пакет, в котором вызывается код, отвечающий за рендеринг в отдельном процессе.
    Рендеринг является CPU-bound задачей, поэтому правильнее
    запускать его через отдельный процесс, чтобы он не воздействовал на главный процесс приложения.
    \item \textbf{repository} -- в этом пакете находится реализация интерфейсов из \textbf{domain}.
\end{enumerate}

\section*{\Large{Примеры кода}}
\addcontentsline{toc}{section}{Примеры кода}

\begin{lstlisting}[language=Python, caption=main.py, captionpos=b]
from app import startup_app

if __name__ == '__main__':
    from aiohttp import web

    web.run_app(startup_app(), host="0.0.0.0", port=8080)
\end{lstlisting}
