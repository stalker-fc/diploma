\section*{\Large{Требования}}
\addcontentsline{toc}{section}{Требования}

В данном пункте кратко перечислим те функциональные и нефункциональные требования, с которыми пришлось столкнуться
в рамках реализации тайлового сервера.

\subsection*{\large{Функциональные требования}}
\addcontentsline{toc}{subsection}{Функциональные требования}

В рамках функциональных требований к сервису можно выделить следующее:
\begin{enumerate}
    \item API должен предусматривать получение тайла по четырем параметрам:
    \begin{itemize}
        \item \textbf{site\_plan\_id} -- идентификатор отображаемого генплана.
        \item \textbf{x} -- координаты тайла по оси X.
        \item \textbf{y} -- координата тайла по оси Y.
        \item \textbf{zoom\_level} -- уровень отдаления карты.
    \end{itemize}
    \item Формат векторных тайлов должен удовлетворять спецификации \textit{Mapbox Vector Tile 2.1}
    \item Предусмотреть механизм кэширования отрендеренных тайлов.
    \item Предусмотреть механизм инвалидации кэша тайлового сервера.
\end{enumerate}

\subsection*{\large{Нефункциональные требования}}
\addcontentsline{toc}{subsection}{Нефункциональные требования}

Из нефункциональных требований выделим следующие:
\begin{enumerate}
    \item Язык программирования \textbf{Python 3.8}
    \item Получение данных из базы данных \textbf{PostgreSQL 12}
    \item Развертывание сервиса осуществлять с помощью \textbf{Docker}
\end{enumerate}
