\section*{\Large{Требования к системе}}
\addcontentsline{toc}{section}{Требования к системе}

В данном пункте кратко перечислим те функциональные и нефункциональные требования, с которыми пришлось столкнуться
в рамках реализации расчетного модуля. Так как объем отчета ограничен, то в целях экономии места, приведем их в менее
формальном виде.

\subsection*{\large{Функциональные требования}}
\addcontentsline{toc}{subsection}{Функциональные требования}

У нас есть 3 группы пользователей, которые будут взаимодействовать с системой.
И каждая из этих групп пользователей имеет свои интересы.

\begin{enumerate}
    \item Технические эксперты, представители заказчика, руководитель проекта
    \begin{itemize}
        \item хотят иметь интерактивный доступ к результатам исследований на различных кейсах,
        \item хотят иметь возможность сравнить результаты работы различных методик решения задачи,
        \item хотят иметь возможность интерактивно запустить алгоритмы на своих данных и получить результат.
    \end{itemize}
    \item Аналитики
    \begin{itemize}
        \item хотят оперативно видеть результаты работы команды исследователей,
        \item хотят проводить анализ результатов различных методик, полученных на разных этапах развития проекта.
    \end{itemize}
    \item Исследователи
    \begin{itemize}
        \item хотят иметь легкий доступ к данным, полученным от пользователей,
        \item хотят иметь простой способ для отправки результатов новой методики для
дальнейшего анализа команде аналитиков,
        \item хотят иметь возможность оперативно добавлять новые методики в действующий
функционал системы.
    \end{itemize}
\end{enumerate}

\subsection*{\large{Нефункциональные требования}}
\addcontentsline{toc}{subsection}{Нефункциональные требования}

Из нефункциональных требований отметим наиболее важные, которые определяют сам процесс проектирования системы
и выбора технологий для реализации.

\begin{enumerate}
    \item \textit{Вычислительные ресурсы}. Для заказчика очень важна сохранность данных,
    которые он предоставляет, поэтому одним из требований является использование собственных серверов компании,
    а не использование облачных решений, популярных в текущий момент в индустрии.

    \item \textit{Уровень анализа предметной области}. Главной сложностью исследовательских проектов является
    невозможность проработки предметной области на достаточно глубоком уровне, чтобы спроектировать систему в
    детальном виде, в связи с короткими сроками и отсутствием качественной экспертизы у заказчика. Необходимо
    предусмотреть высокий уровень гибкости системы, так как требования к ней могут меняться очень часто на противоположные.

    \item \textit{Требования к отчетной документации}. Самой главной ценностью в проекте являются алгоритмы. Именно их и
    требует заказчик, как основной результат нашей деятельности. Алгоритмы должны представлять собой отдельную библиотеку,
    которая должна быть версионируема и быть готова передана заказчику по первому требованию.

    \item \textit{Временные ресурсы}. Основной задачей проекта является показать, что применяемый набор математических
    методов перспективен в плане развития. Эту перспективность требуется показать в достаточно сжатые сроки. Помимо
    команды исследователей, это затрагивает и команду разработки, так как требуется обеспечить качественную визуализацию
    результатов исследований для оперативного уточнения требований.
\end{enumerate}
