\section*{\Large{Постановка задачи}}
\addcontentsline{toc}{section}{Постановка задачи}

% Так как генплан дофига сложный то нужна типа система, которая сможет помочь проектировщикам в их нелегком деле
% Но чтобы создать эту систему нужно дофига усилий

Процесс проектирования генерального плана площадного объекта является непростым
и требует большого объема знаний от специалиста, как в сфере государственного законодательства,
так и сфере правил эксплуатации сооружений.
Из-за большого количества объектов используемых на генплане, специалисту легко допустить ошибку,
которая может быть замечена лишь на поздних этапах проекта.

Самым главным критерием, которым руководствуются инженеры-проектировщики при размещении объекта на генплане -- это
набор допустимых расстояний. Главное, чтобы объект находился достаточно далеко от остальных, чтобы
удовлетворять всем нормам, используемым в строительстве. Дальше уже учитываются технологические ограничения самого
объекта. Например, для трубопроводов это радиусы поворота, количество запорной арматуры, диаметр и т.д.

У каждого спроектированного решения есть своя стоимость, как капитальных затрат, так и эксплуатационных.
В силу того, что CAD-системы не обладают встроенной системой стоимостного моделирования. Инженер не может заранее
узнать, насколько его решение экономически эффективно, так как весь расчёт затрат происходит уже на этапе
составления сметы, чем занимается совсем другой отдел.

Таким образом рынок нуждается, как в системе способной формировать генеральные планы в автоматическом режиме,
так и верифицировать существующие решения в соответствии с набором правил, а также уметь показывать, насколько
экономически обоснованно то или иное решение в сравнении с другими.

Создание инструмента для аналитики и формирования генеральных планов площадных объектов позволит снизить
капитальные затраты при строительстве промышленных объектов благодаря повышению качества проектирования.
Подобный инструмент позволит сравнить несколько различных вариантов планировок одного промышленного объекта,
чтобы впоследствии выбрать оптимальный по ряду критериев, а также получить обоснование,
почему выбранный вариант планировки действительно является оптимальным.

Получение генерального плана площадного объекта в автоматическом режиме на основе
заданного перечня сооружений, технологических и инженерных связей,
формы и размера участка местности, доступного для застройки, а также требований нормативной документации является очень
сложной алгоритмической задачей для которой отсутствуют готовые методики решения.

Создание подобной системы требует проведения массы исследований в сфере алгоритмов
и может занимать не один год. Процесс исследований напрямую связан с взаимодействием с техническими экспертами
в области проектирования генеральных планов.
Для достижения высоких результатов в сфере исследований очень важно оперативное получение обратной связи.
Поэтому остро стоит вопрос по созданию системы, которая сможет сделать процесс исследований наиболее продуктивным.

Но её решение позволит получить алгоритмы более высокого качества,
тем самым уменьшив сроки начала промышленной эксплуатации системы
для решения настоящих промышленных задач.


\textit{Целью} данной работы является создание программного компонента,
который сможет облегчить взаимодействие технических экспертов
и группы исследователей для проведения научных изысканий
в области автоматического проектирования генеральных планов площадных объектов.

Исходя данной цели можно выделить следующие \textit{задачи}:
\begin{itemize}
    \item собрать требования у пользователей системы,
    \item проанализировать возможную нагрузку и вариативность используемых данных,
    \item представить системную и программную архитектуру,
    \item реализовать получившееся решение.
\end{itemize}
