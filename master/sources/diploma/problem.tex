\section*{\Large{Постановка задачи}}
\addcontentsline{toc}{section}{\Large{Постановка задачи}}

% Так как генплан дофига сложный то нужна типа система, которая сможет помочь проектировщикам в их нелегком деле
% Но чтобы создать эту систему нужно дофига усилий

Процесс проектирования генерального плана площадного объекта является непростым
и требует большого объема знаний от специалиста, как в сфере государственного законодательства,
так и сфере правил эксплуатации сооружений.
Из-за большого количества объектов, используемых на генплане, специалисту легко допустить ошибку,
которая может быть замечена лишь на поздних этапах проекта.

Набор правил минимальных допустимых расстояний между объектами является одним из основных критериев,
которым руководствуются инженеры-проектировщики при размещении объекта на генплане.
Основное требование заключается в нахождении объекта на достаточном удалении от остальных,
с целью снижения риска для здоровья и жизни людей во время эксплуатации площадного объекта.
Примером подобных правил являются
требования к ограничению распространения пожара. \cite{Fire}
Они вычисляются на основе класса конструктивной пожарной опасности и степени огнестойкости.
Но для предотвращения распространения пожара также важны
активные средства, какими являются пожарные гидранты, количество и производительность которых определяется
дополнительно строительным объемом сооружений.

Также при проектировании важно учитывать различные технические ограничения используемых объектов.
Например, для трубопроводов таковыми являются диаметр, радиус поворота и пр.

У каждого проектного решения есть стоимость, включающая в себя как капитальные затраты, так и эксплуатационные.
CAD-системы не обладают встроенным инструментом стоимостного моделирования,
поэтому инженер не может заранее узнать,
насколько его решение экономически эффективно для площадного объекта в целом,
так как полноценный расчёт затрат происходит позже на этапе формирования сметы
и оказывается несколько отделенным от процесса проектирования.

Таким образом, рынок нуждается в системе способной как формировать генеральные планы в автоматическом режиме,
так и верифицировать существующие решения в соответствии с заданным набором правил,
а также способной оценивать и сравнивать решения с точки зрения экономической эффективности.

Создание инструмента для аналитики и формирования генпланов площадных объектов позволит снизить
капитальные затраты при строительстве промышленных объектов благодаря повышению качества проектирования.
Подобный инструмент позволит сравнить несколько различных вариантов планировок промышленного объекта,
чтобы выбрать оптимальный по ряду критериев, а также получить обоснование,
почему выбранный вариант планировки действительно является оптимальным.

Формирование генплана площадного объекта в автоматическом режиме
на основе заданного перечня сооружений, технологических связей,
формы и размера участка местности, доступного для застройки,
а также требований нормативной документации является
крайне сложной алгоритмической задачей, для которой отсутствуют готовые методики решения.

Процесс создания системы для решения подобной задачи требует
проведения массы исследований в сфере алгоритмов и может занимать не один год.
В силу столь длительных сроков проведения научных изысканий остро стоит вопрос по созданию инструмента,
способного упростить данный процесс.
Процесс исследований напрямую связан с взаимодействием с техническими экспертами
в области проектирования генеральных планов.
Для достижения высоких результатов в сфере исследований очень важно получение обратной связи,
качество и оперативность которой позволяет сделать процесс исследований максимально продуктивным.

\textit{Целью} данной работы является
упрощение процесса проведения научных изысканий
в области автоматического формирования генеральных планов площадных объектов
путём создания программного компонента.

Исходя данной цели можно выделить следующие \textit{задачи}:
\begin{itemize}
    \item сбор требований пользователей системы,
    \item анализ возможной нагрузки и вариативности используемых данных,
    \item формирование системной и программной архитектуры,
    \item реализация получившегося решения.
\end{itemize}
