\section*{\Large{Постановка задачи}}
\addcontentsline{toc}{section}{Постановка задачи}

% Так как генплан дофига сложный то нужна типа система, которая сможет помочь проектировщикам в их нелегком деле
% Но чтобы создать эту систему нужно дофига усилий




Создание системы подобного класса требует проведения массы научных исследований в сфере алгоритмов.
Но алгоритмы работают с математическими абстракциями.
А инженеры с конкретными объектами, такими как трубопроводы, линии электропередач и прочее.
Процесс обмена опыта между исследователями и инженерами это крайне непростое мероприятие,
требующее механизмов преобразования математических абстракций в объекты реального мира.

Получение качественной обратной связи позволяет достигать высоких результатов в исследованиях, а оперативность
позволяет это делать в короткие сроки.


Создание инструмента для аналитики и формирования генеральных планов площадных объектов позволит снизить
капитальные затраты при строительстве промышленных объектов благодаря повышению качества проектирования.
Подобный инструмент позволит сравнить несколько различных вариантов планировок одного промышленного объекта,
чтобы впоследствии выбрать оптимальный по ряду критериев, а также получить обоснование,
почему выбранный вариант планировки действительно является оптимальным.

Получение генерального плана площадного объекта в автоматическом режиме на основе
заданного перечня сооружений, технологических и инженерных связей,
формы и размера участка местности, доступного для застройки, а также требований нормативной документации является очень
сложной алгоритмической задачей для которой отсутствуют готовые методики решения.

Чтобы создать систему, способную в автоматическом режиме строить генеральные планы площадных объектов,
требуется провести ряд научных исследований в области алгоритмов.
Процесс исследований напрямую связан с взаимодействием с техническими экспертами
в области проектирования генеральных планов.

\textit{Целью} данной работы является создание программного компонента,
который сможет облегчить взаимодействие технических экспертов
и группы исследователей для проведения научных изысканий.
Это позволит получить алгоритмы более высокого качества,
тем самым уменьшив сроки начала промышленной эксплуатации системы
для решения настоящих промышленных задач.

Исходя данной цели можно выделить следующие \textit{задачи}:
\begin{itemize}
    \item cпроектировать модель данных для обмена между всеми компонентами системы,
    \item cпроектировать и реализовать расчетный модуль системы.
\end{itemize}
