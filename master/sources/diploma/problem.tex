\section*{\Large{Постановка задачи}}
\addcontentsline{toc}{section}{Постановка задачи}


Для построения генерального плана сооружения в автоматическом режиме


Основной задачей системы является получения генерального плана площадного объекта в автоматическом режиме на основе
заданного перечня сооружений, технологических и инженерных связей,
формы и размера участка местности, доступного для застройки, а также требований нормативной документации.

Создание инструмента для аналитики и формирования генеральных планов площадных объектов позволит снизить
капитальные затраты при строительстве промышленных объектов благодаря повышению качества проектирования.
Подобный инструмент позволит сравнить несколько различных вариантов планировок одного промышленного объекта,
чтобы впоследствии выбрать оптимальный по ряду критериев, а также получить обоснование,
почему выбранный вариант планировки действительно является оптимальным.

Данная задача является чрезвычайно сложной в алгоритмическом плане и не имеет готовых методик решения.
Для её решения требуется провести ряд научных исследований в области алгоритмов.
Процесс исследований напрямую связан с взаимодействием с техническими экспертами
в области проектирования генеральных планов.




Целью данной работы является создание программного компонента,
который сможет облегчить взаимодействие технических экспертов
и группы исследователей для проведения научных изысканий.
Это позволит получить алгоритмы более высокого качества,
тем самым уменьшив сроки начала промышленной эксплуатации системы для решения настоящих промышленных задач.