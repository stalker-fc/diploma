\section*{\Large{Введение}}
\addcontentsline{toc}{section}{Введение}
Основной задачей системы является получения генерального плана площадного объекта в
автоматическом режиме на основе заданного перечня сооружений, технологических и
инженерных связей, формы и размера участка местности, доступного для застройки, а также
требований нормативной документации.
Создание инструмента для аналитики и формирования генеральных планов площадных объектов
позволит снизить капитальные затраты при строительстве промышленных объектов благодаря
повышению качества проектирования. Подобный инструмент позволит сравнить несколько
различных вариантов планировок одного промышленного объекта, чтобы впоследствии выбрать
оптимальный по ряду критериев, а также получить обоснование, почему выбранный вариант
планировки действительно является оптимальным.
Данная задача является чрезвычайно сложной в алгоритмическом плане и не имеет готовых
методик решения. Для её решения требуется провести ряд научных исследований в области
алгоритмов. Процесс исследований напрямую связан с взаимодействием с техническими
экспертами в области проектирования генеральных планов.
Целью данной работы является создание программного компонента, который сможет облегчить
взаимодействие технических экспертов и группы исследователей для проведения научных
изысканий. Это позволит получить алгоритмы более высокого качества, тем самым уменьшив
сроки начала промышленной эксплуатации системы для решения настоящих промышленных
задач.





При работе с картами на различных устройствах, мы очень редко задумываемся о том, как же, на самом деле,
работает отображение геоданных. Какие технологии используются, чтобы мы смогли увидеть картинку у себя на экране
монитора.

Способ отображения карт можно разбить на 2 больших класса: рендеринг на стороне клиента и рендеринг на стороне
сервера.
В первом случае, в качестве ответа приходят необработанные геоданные и рендеринг осуществляется на машине клиента.
Во втором случае, клиенту уже приходят полностью подготовленные для отображения данных,
которые клиенту следует только отобразить,
никаких математических вычислений ему производить не нужно.

Эти два подхода используются в разных случаях, в зависимости от задач.
При отображении большого объема геоданных следует выбирать подход с рендерингом
на стороне сервера. Это позволит снизить количество данных, передаваемых по сети, снизить нагрузку на
машину клиента, так как не требуется преобразования географических координат в координаты на экране.

\begin{wrapfigure}{r}[0pt]{0.4\textwidth}
    \begin{center}
        \includegraphics[width=\textwidth]{images/introduction/1}
    \end{center}
    \caption{Пример результата работы}
    \label{pic:problem__site-plan}
\end{wrapfigure}
Отрендеренные на сервере геоданные называются тайлы. А сам сервер -- тайловым сервером.
Тайлы можно разбить на 2 класса: растровые и векторные. Растровые тайлы используются в качестве
подложки карты(спутниковые снимки и пр.). Их применяют там, где не нужна интерактивность взаимодействия данных
с пользователем. Векторные тайлы -- это подготовленный кусочек геоданных большого объема. Их используют там, где
требуется интерактивность работы с пользователем. Например: пользователь навел на объект мышкой и этот объект подсветился.

\vskip 2mm
У нас, как раз, случай, с отображением большого количества геоданных.
Генеральный план площадного объекта содержит большое количество различных зданий, коммуникаций и других объектов
необходимых для функционирования этого площадного объекта(см. рис\ref{pic:problem__site-plan}). И так как мы хотим дать
пользователю взаимодействовать с геоданными, то векторные тайлы позволят решить нам нашу проблему.



