\subsection{\large{Разработка математической библиотеки}}
\addcontentsline{toc}{subsection}

В главе с проектированием системы приведена абстрактная архитектура математической библиотеки
и описание общих принципов реализации математических методов. Ниже приведем конкретное воплощение этих общих принципов.

Математическая библиотека называется \textbf{nd\_plan} и имеет следующую структуру пакетов.

\dirtree{%
.1 nd\_plan.
.2 core.
.2 services.
.3 aranei.
.3 barrier.
.3 blast.
.3 buddha.
.3 camino.
.3 commune.
.3 derivable.
.3 distance.
.3 force.
.3 grupper.
.3 gruppo.
.3 heat.
.3 metropole.
.3 netto.
.3 offsets.
.3 optimization.
.3 quadricula.
.3 roadeo.
.3 stater.
.3 strada.
.3 wall.
}

Пакет \textit{core} содержит методы и классы, которые могут быть свободно переиспользованы.
Каждый пакет расположеннный в пакете \textit{services} это реализация отдельной математической методики.

Ниже приведем пример реализации
методики по расчёту радиусов теплового излучения от факелов. Сервис, в котором реализована эта
методика имеет название \textit{heat}. Данный сервис имеет простую методику, поэтому для его работы достаточно
трех обязательных модулей.

\dirtree{%
.1 heat.
.2 \_\_init\_\_.py.
.2 calculate.py.
.2 configuration.py.
.2 model.py.
}

\lstinputlisting[language=Python]{implementation/listings/math/calculate.py}

\lstinputlisting[language=Python]{implementation/listings/math/model.py}

\lstinputlisting[language=Python]{implementation/listings/math/configuration.py}

Так как мы знаем, что расчет теплового излучения может быть произведен только для факелов, то все сооружения,
используемые в этом сервисе, мы можем сразу же обозначить, как факелы. Такое название класса сразу позволит
понять, с какими объектами оперирует методика, не заглядывая в документацию.