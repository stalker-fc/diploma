\subsection{\large{Разработка сервиса по запуску математических методов}}
\addcontentsline{toc}{subsection}


Процесс запуска математического метода можно представить на следующей диаграмме последовательности.






Основными стандартами разрабоотки веб-приложений на языке Python является использование
библиотеки \textit{asyncio}. Для запуска CPU-bound задач используется класс
\textit{concurrent.futures.ProcessPoolExecutor}.

Для запуска используется следующая конструкция.

\begin{lstlisting}
    result = await loop.run_in_executor(
        process_pool_executor,
        execute_task,
        execution_config,
        task.task_id
    )
\end{lstlisting}

Но из-за того что исполняемый процесс является крайне длительным, то нужно обеспечить механизм его завершения.
При завершении процесса с помощью сигнала SIGTERM повреждает экземпляр ProcessPoolExecutor-а и он не может 
запускать новые процессы. В нашем случае данный способ является непримененимым. Чтобы запуск задач работал
предлагается запуск задач с помощью прямого создания экземпляров класса \textit{multiprocessing.Process}.

\begin{lstlisting}
    args = (
        execution_config,
        task_id,
    )
    process = Process(target=execute_long_task, args=args)

    task_execution_process = TaskExecutionProcess(
        task_id,
        process,
    )

    process.start()

    return task_execution_process
\end{lstlisting}
