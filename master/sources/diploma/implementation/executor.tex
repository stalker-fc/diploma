\subsection{\large{Разработка сервиса по запуску математических методов}}
\addcontentsline{toc}{subsection}

Одним из современных стандартов разработки веб-приложений на языке Python является использование
библиотеки \textit{asyncio}. Обычно при использовании данной библиотеки для запуска задач в отдельном процессе
используется экземпляр класса \textit{concurrent.futures.ProcessPoolExecutor}.

\begin{lstlisting}[caption={Пример запуска с использованием \textit{ProcessPoolExecutor}},captionpos=b]
    result = await loop.run_in_executor(
        process_pool_executor,
        execute_task,
        execution_config,
        task.task_id
    )
\end{lstlisting}
\vskip 8mm

Но так как время выполнения математических методов при расчёте генерального плана является значительным,
то необходимо предусмотреть механизм, который позволит завершать расчёт досрочно при необходимости.

Для завершения процессов операционная система использует механизм сигналов. Но если с помощью сигнала
завершить процесс, созданный экземпляром класс \textit{ProcessPoolExecutor}, то экземпляр больше не сможет
создавать новые процесс и для продолжения работы будет требоваться создание экземпляра класса заново.

Таким образом данный способ является неприменимым в нашем случае.
Чтобы корректно работал механизм отмены предлагается запуск задач
с помощью прямого создания экземпляров класса \textit{multiprocessing.Process}.

\begin{lstlisting}[caption={Пример запуска с использованием \textit{multiprocessing.Process}},captionpos=b]
async def handle_long_cpu_bound_task(
        execution_config: ExecutionConfig,
        task_id: int
) -> TaskExecutionProcess:
    args = (
        execution_config,
        task_id,
    )
    process = Process(
                   target=execute_long_task,
                   args=args
               )

    task_execution_process = TaskExecutionProcess(
        task_id,
        process,
    )

    process.start()

    return task_execution_process
\end{lstlisting}
