\subsection{\large{Выбор архитектуры системы}}
\addcontentsline{toc}{subsection}{Выбор архитектуры системы}

Общий вид системы для автоматического проектирования генеральных планов площадных объектов можно представить в виде
двух независимых друг от друга частей: бизнес-части и расчётной.
Эти части системы схематично изображены на диаграмме ниже(см. рис\ \ref{pic:architecture__system-diagram}).

\begin{figure}[H]
	\hspace*{-2.5 cm}\includegraphics[width=0.6\textwidth, left]{architecture/pictures/common/system}
	\caption{Общий вид системы}
	\label{pic:architecture__system-diagram}
\end{figure}
\vskip 5 mm

Бизнес-часть приложения используется для отображения полученных результатов внешним экспертам со стороны заказчика.
Именно в этой части определены все правила и те объекты, которые используются в экспертной оценке.
В расчётной части системы используются объекты внутренней модели данных.
Эти внутренние объекты никак не могут быть интерпретированы внешними экспертами.

Также стоит отметить, что внутренняя модель данных подвержена сильным изменениям в процессе исследований.
Новые алгоритмы решения задачи могут давать куда более качественный результат, но только с учетом того, что
будут использованы дополнительные структуры данных. В то время как внешняя модель данных относительно стабильна
и представлена объектами целиком и полностью понятными внешним экспертам.

Целью данной работы является проектирование и реализация только расчётной части системы, поэтому на всех остальных
диаграммах представленных в работе рассматривается именно эта часть.

Для удовлетворения описанных в предыдущей главе требований предлагается архитектура системы,
использующая микросервисный подход. Данный подход позволяет разбить систему на отдельные блоки,
каждый из которых будет выполнять только свою задачу и не зависеть от функционала других блоков.
Независимость каждого из сервисов существенно упростит процесс разработки и отладки в силу изолированности
каждой задачи. Помимо этого каждый сервис может быть развернут отдельно от остальных, в зависимости
от требований распределения нагрузки. Микросервисный подход к разработке позволит разрабатывать
автономные и слабосвязанные друг с другом сервисы.

Для решения поставленной задачи предлагается следующая архитектура расчётного модуля
(см. рис\ \ref{pic:architecture__common-component}).

\begin{figure}[H]
	\hspace*{-2.5 cm}\includegraphics[width=\textwidth, left]{architecture/pictures/common/component}
	\caption{Диаграмма компонентов}
	\label{pic:architecture__common-component}
\end{figure}
\vskip 5 mm

\noindent Компоненты представленные на диаграмме:
\begin{itemize}
	\item \textbf{Computational Server} -- сервер для запуска расчётных задач.
	\item \textbf{User Device} -- пользовательское устройство для взаимодействия с расчётным сервером.
	\item \textbf{Web Client} -- веб-клиент для получения данных о расчётных задачах.
	\item \textbf{Plan Design Algorithms} -- математическая библиотека, содержащая алгоритмы
	для построения генеральных планов.
	\item \textbf{Plan Design Model} -- расчётная модель данных.
	\item \textbf{Executor} -- сервис, отвечающий за запуск методов математической библиотеки.
	\item \textbf{Storage} -- сервис, отвечающий за чтение/запись расчётной модели данных.
	\item \textbf{Orchestrator} -- сервис, контролирующий выполнение расчётных задач.
	\item \textbf{Plan Design DB} -- база данных (БД).
\end{itemize}

