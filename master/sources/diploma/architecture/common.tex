\subsection*{\large{Выбор архитектуры системы}}
\addcontentsline{toc}{subsection}{Выбор архитектуры системы}

Общий вид системы для автоматического проектирования генеральных планов площадных объектов можно представить в виде
двух независимых друг от друга частей: бизнес и расчётной.
Эти части системы схематично изображены на диаграмме ниже(см. рис\ \ref{pic:architecture__system-diagram}).

\begin{figure}[H]
	\hspace*{-2.5 cm}\includegraphics[width=0.6\textwidth, left]{architecture/pictures/system}
	\caption{Общий вид системы}
	\label{pic:architecture__system-diagram}
\end{figure}
\vskip 5 mm

Бизнес часть приложения используется для отображения полученных результатов внешним экспертам со стороны заказчика.
Именно в этой части определены все правила и те объекты, которые используются в экспертной оценке.
В расчётной части системы содержится большое количество объектов внутренней модели данных, используемых для повышения
качества получаемого решения. Эти внутренние объекты никак не могут быть интерпретированы внешними экспертами.

Также стоит отметить, что внутренняя модель данных подвержена сильным изменениям в процессе исследований.
Новые алгоритмы решения задачи могут давать куда более качественный результат, но только с учетом того, что
будут использованы дополнительные структуры данных. В то время как внешняя модель данных относительно стабильна
и представлена объектами целиком и полностью понятными внешним экспертам.

Целью данной работы является проектирование и реализация только расчетной части системы, поэтому на всех остальных
диаграммах представленных в работе рассматривается именно эта часть.
Для решения поставленной задачи предлагается следующая архитектура расчетного модуля.

\begin{figure}[H]
	\hspace*{-2.5 cm}\includegraphics[width=\textwidth, left]{architecture/pictures/deployment_diagram}
	\caption{Диаграмма размещения}
	\label{pic:architecture__deployment-diagram}
\end{figure}
\vskip 5 mm

\noindent Компоненты представленные на диаграмме размещения:
\begin{itemize}
	\item \textbf{Computational Server} -- сервер для запуска расчётных задач.
	\item \textbf{User Client} -- пользовательское устройство для взаимодействия с расчётным сервером.
	\item \textbf{Web Client} -- веб-клиент для получения данных о расчётных задачах.
	\item \textbf{Orchestrator}
	\begin{itemize}
		\item API -- компонент, предоставляющий REST-API для взаимодействия с сервисом.
		\item Execution -- компонент, контролирующий расчет задачи. Последовательно вызывает этапы выполнения задачи.
	\end{itemize}
	\item \textbf{Storage} -- сервис, отвечающий за чтение/хранение данных расчётных задач.
	\begin{itemize}
		\item API -- компонент, предоставляющий REST-API для взаимодействия с сервисом.
	\end{itemize}
	\item \textbf{Executor} -- сервис, отвечающий за запуск математических методов.
	\begin{itemize}
		\item API -- компонент, предоставляющий REST-API для взаимодействия с сервисом.
		\item Execution -- компонент, осуществляющий запуск метода в отдельном процессе, обработку и сохранение решения.
		\item Math Modules -- математическая библиотека \textbf{nd\_plan}, в которой находятся математические методы.
	\end{itemize}
	\item \textbf{PostgreSQL} -- база данных (БД), используемая для хранения данных для расчётных задач.
\end{itemize}


Расчет генплана площадного объекта -- это комплексная задача, которая состоит из множества различных,
связанных между собой этапов. Результатом задачи является построенный генеральный план.
\begin{itemize}
	\item Местоположение сооружений.
	\item Коммуникационные сети (линии электропередач, трубопроводы).
	\item Эстакады для прокладки коммуникационных сетей.
	\item Внутриплощадочные проезды.
	\item Местоположение площадного объекта на местности.
	\item Зоны теплового излучения.
	\item Зоны поражения взрывных волн.
	\item Объёмы выемки/отсыпки строительной площадки.
	\item Ограждения групп сооружений.
\end{itemize}

Вычисление всех этих объектов представляет последовательный вызов расчетных методов.
Каждый из этих методов выполняет свою определенную задачу: один рассчитывает местоположения зданий, другой, на основе
местоположений зданий рассчитывает конфигурацию внутриплощадочных проездов. Каждый из методов влияет на
результаты других методов. Поэтому разная последовательность их вызовов может давать кардинально разные результаты.

Расчет каждого метода может продолжаться длительное время (например: больше одного часа). Поэтому очень важно
чтобы результаты расчетов методов для задачи были сохранены. В случае возникновения ошибки в процессе расчета задачи,
нельзя допустить, чтобы результаты успешно пройденных этапов были потеряны. Требуется, чтобы после исправления ошибки
в расчете можно было продолжить выполнение задачи.

Поэтому расчет задачи выполняется в двух отдельных контейнерах: \textit{Task Execution Container} отвечает за оркестрацию
вызываемых методов для расчета, а \textit{Calculation Execution Container} за запуск методов из математической библиотеки
в отдельных процессах.

Для того чтобы снизить временные издержки при разработке, будем упрощать интерфейсы взаимодействия между компонентами.
Так как все данные хранятся в базе данных, то нам потребуется реализовать интерфейс взаимодействия между БД
и программным кодом. Данный интерфейс можно будет переиспользовать в трех контейнерах, что позволит сэкономить время на
разработку.

Так как каждый контейнер -- это отдельный процесс, то при запуске задач мы решаем проблему межпроцессного взаимодействия.
Самый простой интерфейс сетевого взаимодействия между двумя процессами -- это использование таблицы в базе данных.
Мы можем применить данный способ потому что у нас малое количество рассчитываемых задач, поэтому проблемы
с использованием БД, как брокера сообщений, просто не возникает.
