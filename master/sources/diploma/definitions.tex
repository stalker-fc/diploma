\section*{\Large{Определения, обозначения и сокращения}}
\addcontentsline{toc}{section}{\Large{Определения, обозначения и сокращения}}

В данной магистерской диссертации применяют следующие термины с соответствующими определениями.

\textit{Генеральный план (генплан, ГП)} в общем смысле —
проектный документ, на основании которого осуществляется планировка,
застройка, реконструкция и иные виды градостроительного освоения территорий.
Основной частью генерального плана (также называемой собственно генеральным планом)
является масштабное изображение, полученное методом графического наложения чертежа
проектируемого объекта на топографический,
инженерно-топографический или фотографический план территории.


\textit{Площадными объектами капитального строительства (ПО)} в данной работе называются
здания, строения, сооружения, а также объекты, строительство которых не завершено, за исключением некапитальных строений,
сооружений и неотделимых улучшений земельного участка (замощение, покрытие и другие)\cite{CapitalBuilding}.


\textit{Здание} — результат строительства, представляющий собой объемную строительную систему,
имеющую надземную и (или) подземную части, включающую в себя помещения,
сети инженерно-технического обеспечения и системы инженерно-технического обеспечения и предназначенную для проживания и
(или) деятельности людей, размещения производства, хранения продукции или содержания животных\cite{SafetyBuildings}.


\textit{Сооружение} — результат строительства,
представляющий собой объемную, плоскостную или линейную строительную систему,
имеющую наземную, надземную и (или) подземную части, состоящую из несущих,
а в отдельных случаях и ограждающих строительных конструкций и предназначенную для выполнения
производственных процессов различного вида, хранения продукции, временного пребывания людей,
перемещения людей и грузов\cite{SafetyBuildings}.


\textit{Строения} — общее понятие зданий и сооружений\cite{CivilCode}.