\subsection{\large{Особенности выполнения расчётных задач}}
\addcontentsline{toc}{subsection}

Каждый объект на генеральном плане имеет собственную методику расчёта.
Для одних объектов методика расчета предоставляется заказчиком,
для других формируется в результате проведения исследований и экспериментов.

Результат работы ряда методик зависит от результата выполнения предыдущих.
Например, вычисление местоположения ограждений для групп сооружений невозможно без рассчитанного
местоположения сооружений.

В силу сложности задачи построения генплана
возникают типичные проблемы использования сложных математических алгоритмов,
обрабатывающих большой набор входных данных: долгое время выполнения,
высокую нагрузку на центральный процессор и высокий уровень потребления оперативной памяти.
