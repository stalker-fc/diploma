\subsection*{\large{Особенности выполнения расчётных задач}}
\addcontentsline{toc}{subsection}{Особенности выполнения расчётных задач}

Генеральный план площадного объекта состоит из множества объектов.
Каждый из этих объектов имеет собственную методику расчёта.
Методика расчёта для части объектов предоставлена заказчиком,
но для другой части объектов методика расчёта
является результатом исследований и проведения множества экспериментов.

Результат работы ряда методик зависит от выполнения предыдущих.
Например, вычисление местоположения ограждений для групп сооружений невозможно без рассчитанного
местоположения сооружений.

В силу сложности задачи построения генплана
возникают типичные проблемы использования сложных алгоритмов: долгое время выполнения,
высокую нагрузку на центральный процессор и высокий уровень потребления оперативной памяти.

Так как генплан является совокупностью объектов, каждый из которых получен в результате выполнения
собственной методики, и порядок выполнения этих методик строго последовательный, а также выполнение
каждой методики может занимать достаточно продолжительное время, то необходимо предусмотреть
механизм расчёта задачи по этапам, с сохранением промежуточных результатов,
а также возможности продолжить расчёт с того этапа, на котором возникла ошибка.
