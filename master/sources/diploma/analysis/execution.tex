\section{\Large{Особенности выполнения расчётных задач}}
\addcontentsline{toc}{section}{Особенности выполнения расчётных задач}

Генеральный план площадного объекта состоит из множества объектов.
Каждый из этих объектов имеет собственную методику расчёта.
Методика расчёта для части объектов предоставлена заказчиком,
но для другой части объектов методика расчёта
является результатом исследований и проведения множества экспериментов.

Результат работы ряда методик зависит от результата выполнения предыдущих.
Например, вычисление местоположения ограждений для групп сооружений невозможно без рассчитанного
местоположения сооружений.

В силу сложности задачи построения генплана
возникают типичные проблемы использования сложных математических алгоритмов,
способных обрабатывать большой набор входных данных: долгое время выполнения,
высокую нагрузку на центральный процессор и высокий уровень потребления оперативной памяти.
