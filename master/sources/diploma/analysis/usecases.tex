\subsection{\Large{Бизнес-процессы}}
\addcontentsline{toc}{subsection}

Существует множество возможных вариантов использования системы для проведения исследований.
Перечислим некоторые из них.
\begin{itemize}
	\item Добавлен новый расчётный элемент или изменены требования размещения к существующему.
	\item Выбор наиболее эффективных методик расчёта для конкретного кейса.
	\item Редактирование расчётных данных с целью анализа лучших стратегий применения той или иной методики.
\end{itemize}

В виде диаграм представим два наиболее распространенных примера: добавлены новый расчётный элемент
или изменены требования размещения к существующему, выбор наиболее эффективных методик расчёта для конкретного кейса.

Ниже представлена EPC-диаграмма процесса проведения исследований и
внедрения алгоритмической методики(см. рис\ \ref{pic:analysis__usecases-epc}).

\begin{figure}[H]
	\hspace*{-2.5 cm}\includegraphics[width=0.64\textwidth, left]{analysis/pictures/usecases/common_epc}
	\caption{Диаграмма процесса внедрения алгоритмической методики}
	\label{pic:analysis__usecases-epc}
\end{figure}
\vskip 5 mm

Главным инициатором исследований является заказчик. Именно заказчик знает, где найти людей,
обладающих экспертными знаниями в проектировании генеральных планов.
Заказчик так же определяет приоритет появления
новых расчётных элементов на генплане сооружения и количество требований к их размещению.

В любом случае генплан, полученный в автоматическом режиме будет нуждаться в ручной корректировке,
так как в силу очень высокой сложности объектов, невозможно заложить в алгоритмы все требования к формированию.

Процесс обновление алгоритмов выглядит следующим образом.
На первом этапе заказчик решает, требуется ли ему улучшить качество размещения уже существующего элемента
или добавить на генплан новый расчётный элемент.

После этого заказчик сообщает аналитику контактную информацию технического эксперта.
Именно технический эксперт может предоставить презентативные данные для отладки методики аналитику,
которые смогут удовлетворить требованиям заказчика.

После получения данных от технического эксперта аналитик оформляет их в виде доступном для загрузки в расчётный модуль.
После того как данные загружены в расчётный модуль, исследователи могут приступать к выработке решения
правильного размещения заданного объекта генерального плана площадного объекта.

У исследователя всегда есть два варианта решения поставленной задачи: создать новую методику решения
или использовать уже существующую.
После того как сделан выбор, проводится ряд экспериментов,
наиболее удачные результаты экспериментов сохраняются в хранилище и отправляются на рассмотрение аналитику.
Если аналитик делает вывод, что данная методика, позволяет получить решение, соответствующее требованиям
заказчика, то результаты эксперимента уже показываются техническому эксперту со стороны заказчика,
чтобы он дополнительно убедился, что реализация данной методики не противоречит другим требованиям
проектирования генпланов.

Если все хорошо, то методика встраивается в существующее решение.
Если же есть какие-либо недостатки, то они фиксируются и устраняются.

Ниже представлена EPC-диаграмма выбор наиболее эффективных методик расчёта для конкретного кейса
(см. рис\ \ref{pic:analysis__usecases-analytics-epc}).

\begin{figure}[H]
	\hspace*{-2.5 cm}\includegraphics[width=0.55\textwidth, left]{analysis/pictures/usecases/analytics_epc}
	\caption{Диаграмма процесса выбора эффективной методики расчёта}
	\label{pic:analysis__usecases-analytics-epc}
\end{figure}
\vskip 5 mm

Все действия на этой диаграмме выполняются аналитиком в системе автоматического
расчёта генерального плана площадного объекта.
Входные данные задачи определены, на них происходит проведение анализа.
Аналитик выбирает необходимые этапы задачи и методы для проведения исследований, производит расчёт,
анализирует получившееся решение. Если результат решения удовлетворительный,
то расчёт доводится до конца и в результате получается сформированный генплан.