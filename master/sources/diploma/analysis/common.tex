\subsection*{\large{Пользователи системы и их потребности}}
\addcontentsline{toc}{subsection}{Пользователи системы и их потребности}

На текущий момент отсутствует полноценная методика
по автоматическому формированию генеральных планов площадных объектов.
Это означает, что для достижения результата потребуется провести ряд исследований в области алгоритмов
формирования генеральных планов.
Любые исследования для реальных промышленных задач напрямую связаны с активной консультацией с техническими
экспертами со стороны заказчика, а также грамотного оформления всех результатов проведенных исследований
и быстрой возможностью их повторения.

На время научных изысканий система будет являться исследовательским прототипом.
И она должна, как позволять интерактивно отображать полученные результаты,
так и фиксировать все проведенные эксперименты.


Можно выделить три группы пользователей, которые будут взаимодействовать с системой:
\begin{enumerate}
    \item Технические эксперты со стороны заказчика.
    \item Аналитики.
    \item Исследователи.
\end{enumerate}

С каждой из групп пользователей было проведено интервью с целью выяснения их потребностей в работе с системой
и был составлен список этих потребностей, который представлен ниже.

\begin{enumerate}
    \item {
        \textit{Технические эксперты со стороны заказчика}
        \begin{itemize}
            \item хотят иметь интерактивный доступ к результатам исследований на различных кейсах.
        \end{itemize}
    }
    \item {
        \textit{Аналитики}
        \begin{itemize}
            \item хотят оперативно видеть результаты работы команды исследователей,
            \item хотят проводить анализ результатов различных методик, полученных на разных этапах развития проекта.
        \end{itemize}
    }
    \item {
        \textit{Исследователи}
        \begin{itemize}
            \item хотят иметь удобный и простой доступ к данным, полученным от пользователей,
            \item хотят иметь простой способ для отправки результатов новой методики для дальнейшего анализа команде аналитиков,
            \item хотят иметь возможность оперативно добавлять новые методики в действующий функционал системы.
        \end{itemize}
    }
\end{enumerate}

