\section*{\large{Описание и моделирование объекта автоматизации}}
\addcontentsline{toc}{subsection}{Описание и моделирование объекта автоматизации}

1.1. Описание и моделирование объекта автоматизации.................... 8
1.2. Анализ существующей инфраструктуры .................................... 11
1.3. Формирование требований к средству автоматизации..............13

Система проектирования генеральных планов площадных объектов в автоматическом режиме
является исследовательским прототипом, что означает, что она не готова к полноценной промышленной эксплуатации,
в силу того, что еще полноценно не выработаны алгоритмические подходы.

Это ведет к тому, что существующая методика решения задачи может в одночасье стать неактуальной,
потому что найдется новая методика, которая будет давать более качественное решение.
Поэтому система должна быть очень гибкой, чтобы без больших затрат со стороны разработки вносить подобные изменения.

Главными персонажами при решении сложных математических задач являются аналитики и исследователи.
Аналитики собирают требования у заказчика и предоставляют их исследователям



\subsection*{\large{Нефункциональные требования}}
\addcontentsline{toc}{subsection}{Нефункциональные требования}

Из нефункциональных требований отметим наиболее важные, которые определяют сам процесс проектирования системы
и выбора технологий для реализации.

\begin{enumerate}
    \item \textit{Вычислительные ресурсы}. Для заказчика очень важна сохранность данных,
    которые он предоставляет, поэтому одним из требований является использование собственных серверов компании,
    а не использование облачных решений, популярных в текущий момент в индустрии.

    \item \textit{Уровень анализа предметной области}. Главной сложностью исследовательских проектов является
    невозможность проработки предметной области на достаточно глубоком уровне, чтобы спроектировать систему в
    детальном виде, в связи с короткими сроками и отсутствием качественной экспертизы у заказчика. Необходимо
    предусмотреть высокий уровень гибкости системы, так как требования к системе могут измениться на противоположные.

    \item \textit{Требования к отчетной документации}. Самой главной ценностью в проекте являются алгоритмы. Именно их и
    требует заказчик, как основной результат нашей деятельности. Алгоритмы должны представлять собой отдельную библиотеку,
    которая должна быть версионируема и быть готова передана заказчику по первому требованию.

    \item \textit{Временные ресурсы}. Основной задачей проекта является показать, что применяемый набор математических
    методов перспективен в плане развития. Эту перспективность требуется показать в достаточно сжатые сроки. Помимо
    команды исследователей, это затрагивает и команду разработки, так как требуется обеспечить качественную визуализацию
    результатов исследований для оперативного уточнения требований.
\end{enumerate}
