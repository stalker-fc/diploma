\section*{\large{Описание и моделирование объекта автоматизации}}
\addcontentsline{toc}{subsection}{Описание и моделирование объекта автоматизации}

1.1. Описание и моделирование объекта автоматизации.................... 8
1.2. Анализ существующей инфраструктуры .................................... 11
1.3. Формирование требований к средству автоматизации..............13

Система проектирования генеральных планов площадных объектов в автоматическом режиме
является исследовательским прототипом, что означает, что она не готова к полноценной промышленной эксплуатации,
в силу того, что еще полноценно не выработаны алгоритмические подходы.

Это ведет к тому, что существующая методика решения задачи может в одночасье стать неактуальной,
потому что найдется новая методика, которая будет давать более качественное решение.
Поэтому система должна быть очень гибкой, чтобы без больших затрат со стороны разработки вносить подобные изменения.

Главными персонажами при решении сложных математических задач являются аналитики и исследователи.
Аналитики собирают требования у заказчика и предоставляют их исследователям









