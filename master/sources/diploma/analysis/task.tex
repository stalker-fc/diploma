\section{\Large{Предметная область задачи}}
\addcontentsline{toc}{section}{Предметная область задачи}

Задача по автоматической генерации генерального плана площадного объекта является очень сложной не только
в алгоритмическом плане, но еще и в техническом. Для расчёта генплана необходимо учесть большое количество входных
данных.
Ниже представлен пример набора входных данных.
\begin{itemize}
    \item допустимая для строительства область на карте;
    \item стоимостная модель расчета стоимости инженерной подготовки;
    \item перечень сооружений с указанием габаритов (ширина, длина, радиус),
    степени огнестойкости, категории взрывопожарной и пожарной опасности, конструктивной пожарной опасности;
    \item параметры коммуникаций между сооружениями проектируемого объекта;
    \item местоположения внешних точек подключения площадного объекта (дорога, ВЛ, трубопроводы внешнего транспорта и т.п.);
    \item параметры цифровой модели рельефа(ЦМР);
\end{itemize}

Результатом работы системы является генеральный план площадного объекта.
Он, в свою очередь, состоит из множества технических элементов.
Ниже представлен пример объектов, из которых может состоять генплан.
\begin{itemize}
    \item фигуру площадного объекта, содержащей заданные сооружения;
    \item местоположения сооружений;
    \item схемы технологических эстакад минимальной длины, соединяющей все сооружения, введенные пользователем;
    \item схему внутриплощадочных проездов;
    \item стоимость инженерной подготовки, коэффициента застройки территории и длины технологических связей;
    \item зоны распространения теплового потока;
    \item зоны распространения взрывной волны;
\end{itemize}

Все перечисленные выше объекты очень разнородны по своему содержанию и структуре.