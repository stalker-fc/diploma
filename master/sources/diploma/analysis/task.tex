\subsection{\large{Предметная область задачи}}
\addcontentsline{toc}{subsection}

Задача по автоматическому формированию генерального плана площадного объекта
представляет существенные трудности не только в алгоритмическом плане, но еще и в техническом.
Для расчёта генплана необходимо учесть большое количество входных данных.
Ниже представлен пример набора входных данных:
\begin{itemize}
    \item допустимая для строительства область на карте;
    \item стоимостная модель расчета стоимости инженерной подготовки;
    \item перечень сооружений с указанием габаритов (ширина, длина, радиус),
    степени огнестойкости, категорий взрывопожарной и пожарной опасности, конструктивной пожарной опасности;
    \item параметры технологических коммуникаций между сооружениями проектируемого объекта (электрокабели, трубопроводы и т.д.);
    \item местоположения внешних точек подключения площадного объекта
    (дорога, воздушная линия, трубопроводы внешнего транспорта и т.п.);
    \item параметры цифровой модели рельефа;
\end{itemize}

Результатом работы системы является генеральный план площадного объекта.
Он, в свою очередь, состоит из множества технологических элементов.
Ниже представлены примеры объектов, которые могут быть включены в генплан:
\begin{itemize}
    \item фигура площадного объекта, содержащая заданные сооружения;
    \item местоположения сооружений;
    \item схема технологических эстакад минимальной длины, соединяющей все сооружения, заданные пользователем;
    \item схема внутриплощадочных проездов;
    \item стоимость инженерной подготовки, коэффициент застройки территории и длина технологических связей;
    \item зоны распространения теплового потока;
    \item зоны распространения взрывной волны;
\end{itemize}

Все перечисленные выше объекты очень разнородны по своему содержанию и структуре.