\subsection{\large{Формирование требований}}
\addcontentsline{toc}{subsection}

\subsubsection{Функциональные требования}
\addcontentsline{toc}{subsubsection}

На основе анализа сформированы следующие функциональные требования:
\begin{itemize}
    \item {
        Должна присутствовать возможность расчёта генерального плана площадного объекта в автоматическом режиме.
    }
    \item {
        Автоматический расчёт генерального плана площадного объекта должен быть разбит на этапы.
    }
    \item {
        Результат каждого этапа расчёта должен быть сохранён в долговременное хранилище.
    }
    \item {
        Должна присутствовать возможность продолжить расчёт с последнего успешно завершенного этапа.
    }
    \item {
        Должна быть возможность сравнения одинаковых расчётных объектов, полученных с путем применения разных методик.
    }
    \item {
        Должна присутствовать возможность загрузки данных, полученных от технических экспертов, в расчётный модуль.
    }
    \item {
        Должна присутствовать возможность загрузки результатов экспериментов, а также информации об особенностях
        проведения экспериментов в расчётный модуль.
    }
\end{itemize}

\subsubsection{\large{Нефункциональные требования}}
\addcontentsline{toc}{subsubsection}

% https://habr.com/ru/post/231961/
Из нефункциональных требований отметим наиболее важные, которые определяют сам процесс проектирования системы
и выбора технологий для реализации.

\begin{enumerate}
    \item {
        \textit{Вычислительные ресурсы}.
        Одним из требований заказчика является использование собственных серверов компании.
        Для проведения исследований выделен один вычислительный сервер на операционной системе Ubuntu 20.04 LTS.
    }
    \item {
        \textit{Особенность выполнения расчётов}.
        Вызов алгоритмически сложной части системы должен осуществляться в отдельном процессе.
    }
    \item {
        \textit{Временные ресурсы}.
        Основной задачей проекта является демонстрация перспективности развития применяемого набора математических методов.
        Эту перспективность требуется показать в короткие сроки.
    }
    \item {
        \textit{Гибкость системной архитектуры}.
        Проект является исследовательским, поэтому
        отсутствует достаточно глубокий уровень проработки предметной области для качественного проектирования системы.
        Необходимо предусмотреть высокий уровень гибкости системы,
        так как требования к системе могут оперативно и непредсказуемо изменяться.
    }
    \item {
        \textit{Требования к отчётной документации}.
        Алгоритмы являются основным результатом деятельности в проекте.
        Все разработанные алгоритмы должны быть оформлены в отдельную библиотеку.
        Библиотека должна быть версионируема.
    }
\end{enumerate}