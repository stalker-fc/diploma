



\subsection*{\large{Нефункциональные требования}}
\addcontentsline{toc}{subsection}{Нефункциональные требования}

Из нефункциональных требований отметим наиболее важные, которые определяют сам процесс проектирования системы
и выбора технологий для реализации.

\begin{enumerate}
    \item {
        \textit{Вычислительные ресурсы}.
        Одним из требований заказчика является использование собственных серверов компании.
        Для проведения исследований выделен один вычислительный сервер.
    }
    \item {
        \textit{Временные ресурсы}.
        Основной задачей проекта является показать, что применяемый набор математических
        методов перспективен в плане развития. Эту перспективность требуется показать в короткие сроки.
    }
    \item {
        \textit{Гибкость системной архитектуры}.
        Проект является исследовательским, поэтому
        отсутствует достаточно глубокий уровень проработки предметной области для качественного проектирования системы.
        Необходимо предусмотреть высокий уровень гибкости системы,
        так как требования к системе могут оперативно и непредсказуемо изменяться.
    }
    \item {
        \textit{Требования к отчетной документации}.
        Алгоритмы являются основным результатом деятельности в проекте.
        Все разработанные алгоритмы должны нахожиться в отдельной библиотеке.
        Библиотека должна быть версионируема.
    }
\end{enumerate}