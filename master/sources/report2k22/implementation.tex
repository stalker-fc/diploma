\section*{\Large{Реализация}}
\addcontentsline{toc}{section}{Реализация}

При реализации описанной архитектуры была получена следующая структура проекта.
\dirtree{%
.1 /.
.2 app.
.3 api.
.4 \_\_init\_\_.py.
.4 controller.py.
.4 exception\_handlers.py.
.4 middlewares.py.
.4 model.py.
.4 responses.py.
.4 serialization.py.
.4 views.py.
.3 data.
.4 mappers.
.5 services.
.6 services mappings implementations.
.5 \_\_init\_\_.py.
.5 registry.py.
.5 utils.py.
.4 \_\_init\_\_.py.
.4 data\_manager.py.
.4 features\_registry.py.
.4 repository.py.
.4 storage.py.
.3 domain.
.4 \_\_init\_\_.py.
.4 listener.py.
.4 model.py.
.4 queue.py.
.4 repository.py.
.3 execution.
.4 \_\_init\_\_.py.
.4 executor.py.
.4 handler.py.
.4 listener.py.
.4 queue.py.
.4 storage.py.
.3 \_\_init\_\_.py.
.3 app.py.
.3 exceptions.py.
.3 logger.py.
.2 Dockerfile.
.2 README.md.
.2 main.py.
.2 requirements.txt.
}

Само приложение находится в директории \textbf{app}.
Всего получилось 4 python-пакета, в которых и скрыта основная логика работы.
\begin{enumerate}
    \item \textbf{api} -- реализует API сервисы через HTTP. То есть endpoint-ы,
    модели запросов/ответов к серверу, а также взаимодействие с \textit{ITaskRepository} и \textit{ITaskDataManager}
    \item \textbf{data} -- в этом пакете находится реализация интерфейсов из \textbf{domain}.
    \item \item \textbf{domain} -- пакет, в котором определены только интерфейсы взаимодействия между модулями программы.
    Именно здесь и реализована диаграмма классов (см. рис\ \ref{pic:architecture__execution-classes-diagram}). Непосредственная
    реализация обозначенных интерфейсов находится в других пакетах.
    \item \textbf{execution} -- пакет, в котором вызывается код, отвечающий за запуск математических методов
    в отдельном процессе.
\end{enumerate}

\section*{\Large{Примеры кода}}
\addcontentsline{toc}{section}{Примеры кода}

\begin{lstlisting}[language=Python, caption=main.py, captionpos=b]
from app import startup_app
import uvicorn

if __name__ == '__main__':
    app = startup_app()
    uvicorn.run(app, host="0.0.0.0", port=8080)
\end{lstlisting}

\vskip 10 mm
\begin{lstlisting}[caption=Dockerfile, captionpos=b]]
FROM python:3.8@sha256:4c4e6735f46e7727965d1523015874ab08f71377b3536b8789ee5742fc737059

WORKDIR /app

ENV LC_ALL C.UTF-8
ENV LANG C.UTF-8
ENV N_WORKERS 8

COPY requirements.txt .
RUN pip3 install --no-cache-dir -r requirements.txt

RUN pip3 check

COPY main.py .

COPY /app ./app

ENTRYPOINT /bin/bash -c "gunicorn run_app:app --workers=${N_WORKERS} --bind 0.0.0.0:8080 --worker-class aiohttp.GunicornWebWorker --timeout 0"
\end{lstlisting}

\vskip 10 mm
\begin{lstlisting}[language=Python, caption=domain/model.py, captionpos=b]
from dataclasses import dataclass
from enum import Enum
from typing import Optional
from uuid import UUID

from dataclasses_json import dataclass_json

from nd_plan.interfaces import MethodName


class TaskStatus(Enum):
    CREATED = "CREATED"
    QUEUED = "QUEUED"
    RUNNING = "RUNNING"
    CANCELLED = "CANCELLED"
    SUCCESS = "SUCCESS"
    FAILURE = "FAILURE"


@dataclass_json
@dataclass
class Task:
    task_id: UUID
    status: TaskStatus
    method_name: nd_plan.MethodName
    input_data: nd_plan.InputDataType
    configuration: Optional[nd_plan.ConfigurationType] = None
    output_data: Optional[nd_plan.SolutionType] = None
\end{lstlisting}

\vskip 10 mm
