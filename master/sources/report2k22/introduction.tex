\section*{\Large{ВВЕДЕНИЕ}}
\addcontentsline{toc}{section}{ВВЕДЕНИЕ}
Автоматическое формирование генерального плана (ГП) площадного объекта капитального строительства является чрезвычайно
сложной задачей, как в алгоритмическом, так и в технологическом плане.
На генеральном плане помимо сооружений отражены внутриплощадочные проезды, различные трубопроводы, линии электропередач,
технологические эстакады, пожарные гидранты и прочие объекты,
необходимые для функционирования того или иного площадного объекта.

Для каждого объекта ГП заданы определенные требования к его размещению.
Так как объектов много и они очень разнообразны по своей структуре,
то использовать единый алгоритм для их расстановки невозможно.

Каждый объект на генеральном плане имеет собственную методику расчёта.
Для одних объектов методика расчета предоставляется заказчиком,
для других формируется в результате проведения исследований и экспериментов.

Результат работы ряда методик зависит от результата выполнения предыдущих.
Например, вычисление местоположения ограждений для групп сооружений невозможно без рассчитанного
местоположения сооружений.
