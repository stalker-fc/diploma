\section*{\Large{Требования}}
\addcontentsline{toc}{section}{Требования}

Ниже перечислены те функциональные и нефункциональные требования
для сервиса запуска математических методов библиотеки \textbf{nd\_plan}.

\subsection*{\large{Функциональные требования}}
\addcontentsline{toc}{subsection}{Функциональные требования}

В рамках функциональных требований к сервису можно выделить следующее:
\begin{enumerate}
    \item {
       Должна присутствовать возможность расчёта генерального плана площадного объекта в автоматическом режиме.
    }
    \item {
        Автоматический расчёт генерального плана площадного объекта должен быть разбит на этапы.
    }
    \item {
        Результат каждого этапа расчёта должен быть сохранён в долговременное хранилище.
    }
    \item {
        Должна присутствовать возможность продолжить расчёт с последнего успешно завершенного этапа.
    }
    \item API должен придерживаться концепции REST.
    \item API должен предусматривать асинхронный запуск расчётной задачи.
    \item API должен использовать JSON в качестве обмена данных с клиентом.
\end{enumerate}

\subsection*{\large{Нефункциональные требования}}
\addcontentsline{toc}{subsection}{Нефункциональные требования}

Из нефункциональных требований выделим:
\begin{enumerate}
    \item Язык программирования \textbf{Python 3.8}
    \item Развертывание сервиса осуществлять с помощью \textbf{Docker}
\end{enumerate}
