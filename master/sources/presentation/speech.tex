%! Author = sergey
%! Date = 18.05.2022

% Preamble
\documentclass[11pt]{article}

% Packages
\usepackage{amsmath}
\usepackage[english,russian]{babel}

% Document
\begin{document}

1. ~ 18 сек

Здравствуйте, уважаемая комиссия!

Меня зовут Степанов Сергей и сегодня я хочу Вам рассказать о результатах,
полученных в рамках работы над магистерской диссертацией по теме:
Проектирование и реализация расчётного модуля для системы автоматического формирования
генеральных планов площадных объектов капитального строительства.


2. ~ 1 мин 30 сек ОЧЕНЬ МНОГО!

Генеральным планом называется
проектный документ, на основании которого осуществляется планировка,
застройка, реконструкция и иные виды градостроительного освоения территорий.
Площадными объектами капитального строительства в данной работе называются здания, строения, сооружения.

На основе генерального плана проводится оценка, как капитальных, так и эксплуатационных затрат.
За проектирование генпланов крупных технологических объектов отвечают проектные институты,
а в особенности инженеры-проектировщики. Их главным рабочим инструментом
являются системы автоматизированного проектирования или САПР.

САПР не обладают полноценным инструментом стоимостного моделирования,
поэтому инженер не может заранее узнать,
насколько его решение экономически эффективно для проекта в целом,
так как расчёт затрат происходит позже на этапе формирования сметы.

В силу высокой сложности проектирования генпланов,
рынок нуждается в системе способной как формировать генеральные планы в автоматическом режиме,
так и верифицировать существующие решения в соответствии с заданным набором правил,
а также способной оценивать и сравнивать решения с точки зрения экономической эффективности.

Формирование генплана является крайне сложной алгоритмической задачей,
для которой отсутствуют готовые методики решения.

Процесс создания системы для решения подобной задачи требует
проведения массы исследований в сфере алгоритмов и может занимать не один год.
Процесс исследований напрямую связан с взаимодействием с техническими экспертами
в области проектирования генеральных планов.
Для достижения высоких результатов в сфере исследований очень важно получение обратной связи,
качество и оперативность которой позволяет сделать процесс исследований максимально продуктивным.

3. ~ 10 сек

Целью данной работы является
упрощение процесса проведения научных изысканий
в области автоматического формирования генеральных планов площадных объектов
путём создания программного компонента.

4. ~ 16 сек
Исходя данной цели можно выделить следующие \textit{задачи}:
\begin{itemize}
    \item сбор и анализ требований пользователей системы,
    \item анализ возможной нагрузки и вариативности используемых данных,
    \item формирование системной и программной архитектуры,
    \item реализация полученного решения.
\end{itemize}


\end{document}