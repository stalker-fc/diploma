%! Author = sergey
%! Date = 18.05.2022

% Preamble
\documentclass[11pt]{article}

% Packages
\usepackage{amsmath}
\usepackage[english,russian]{babel}

% Document
\begin{document}


\paragraph{Титульный слайд}.

Здравствуйте, уважаемая комиссия!

Меня зовут Степанов Сергей и сегодня я хочу Вам рассказать о результатах,
полученных в рамках работы над магистерской диссертацией по теме:
Проектирование и реализация расчётного модуля для системы автоматического формирования
генеральных планов площадных объектов капитального строительства.


\paragraph{Введение}.

\textit{Генеральный план (генплан, ГП)} в общем смысле —
проектный документ, на основании которого осуществляется планировка,
застройка, реконструкция и иные виды градостроительного освоения территорий.

Описанное выше применимо и для сферы строительства.
Любому сложному строительству предшествует этап проектирования.
Результатом этого этапа является генеральный план площадного объекта .
Именно на его основе проводится оценка, как капитальных, так и эксплуатационных затрат.
За проектирование крупных технологических объектов отвечают проектные институты,
а также инженеры-проектировщики, которые в них работают.




\paragraph{Цель работы}.

\textit{Целью} данной работы является
упрощение процесса проведения научных изысканий
в области автоматического формирования генеральных планов площадных объектов
путём создания программного компонента.

\paragraph{Задачи}.


\end{document}