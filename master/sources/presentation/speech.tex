%! Author = sergey
%! Date = 18.05.2022

% Preamble
\documentclass[11pt]{article}

% Packages
\usepackage{amsmath}
\usepackage[english,russian]{babel}

% Document
\begin{document}

    1. ~ 18 сек

    Здравствуйте, уважаемая комиссия!

    Меня зовут Степанов Сергей и сегодня я хочу Вам рассказать о результатах,
    полученных в рамках работы над магистерской диссертацией по теме:
    Проектирование и реализация расчётного модуля для системы автоматического формирования
    генеральных планов площадных объектов капитального строительства.


    2. ~ 1 мин 30 сек ОЧЕНЬ МНОГО!

    Генеральным планом называется
    проектный документ, на основании которого осуществляется планировка,
    застройка, реконструкция и иные виды градостроительного освоения территорий.
    Площадными объектами капитального строительства в данной работе называются здания, строения, сооружения.

    На основе генерального плана проводится оценка, как капитальных, так и эксплуатационных затрат.
    За проектирование генпланов крупных технологических объектов отвечают проектные институты,
    а в особенности инженеры-проектировщики. Их главным рабочим инструментом
    являются системы автоматизированного проектирования или САПР.

    САПР не обладают полноценным инструментом стоимостного моделирования,
    поэтому инженер не может заранее узнать,
    насколько его решение экономически эффективно для проекта в целом,
    так как расчёт затрат происходит позже на этапе формирования сметы.

    В силу высокой сложности проектирования генпланов,
    рынок нуждается в системе способной как формировать генеральные планы в автоматическом режиме,
    так и верифицировать существующие решения в соответствии с заданным набором правил,
    а также способной оценивать и сравнивать решения с точки зрения экономической эффективности.

    Формирование генплана является крайне сложной алгоритмической задачей,
    для которой отсутствуют готовые методики решения.

    Процесс создания системы для решения подобной задачи требует
    проведения массы исследований в сфере алгоритмов и может занимать не один год.
    Процесс исследований напрямую связан с взаимодействием с техническими экспертами
    в области проектирования генеральных планов.
    Для достижения высоких результатов в сфере исследований очень важно получение обратной связи,
    качество и оперативность которой позволяет сделать процесс исследований максимально продуктивным.

    3. ~ 10 сек

    Целью данной работы является
    упрощение процесса проведения научных изысканий
    в области автоматического формирования генеральных планов площадных объектов
    путём создания программного компонента.

    4. ~ 16 сек
    Исходя данной цели можно выделить следующие \textit{задачи}:
    \begin{itemize}
        \item сбор и анализ требований пользователей системы,
        \item анализ возможной нагрузки и вариативности используемых данных,
        \item формирование системной и программной архитектуры,
        \item реализация полученного решения.
    \end{itemize}

    5. ~ 0 мин 33 сек

    Можно выделить три группы пользователей, которые будут взаимодействовать с системой:
    \begin{enumerate}
        \item {
            \textit{Технические эксперты.} Они обладают профессиональными знаниями в сфере
            проектирования генеральных планов площадных объектов. Именно на их экспертизе и базируется итоговое
            качество получаемого решения.
        }
        \item{
            \textit{Аналитики.} Они являются связкой между техническими экспертами и исследователями.
            Именно они презентуют полученные результаты техническим экспертам и подготавливают данные в том формате,
            которым могут воспользоваться исследователи.
        }
        \item{
            \textit{Исследователи.} Основной их деятельностью является решение именно математической задачи,
            применение и обоснование методики, которая будет давать наилучший результат.
        }
    \end{enumerate}

    6. ~ X мин X сек

    Есть несколько различных вариантов использований системы, рассмотрим самый распространенный:
    добавлен новый расчётный элемент или изменены требования размещения к существующему.

    На слайде изображена диаграмма процесса проведения исследований и
    внедрения алгоритмической методики. Для оперативности её понимания,
    приведем пример её использования.

    Заказчик попросил строить эстакады трёх различных типов: подземные, наземные, надземные.
    Аналитик обращается к эксперту с просьбой предоставить проектные данные, где есть коммуникации,
    тех типов, которые можно разбить на разные типы эстакад.
    На основе этих данных аналитик формирует кейс в формате системы.
    Далее аналитик ставит задачу исследователям и сообщает, как получить данные кейса.
    Далее начинается процесс исследований в рамках которого и определяется способ получения необходимого результата.

    Если аналитик делает вывод, что данная методика, позволяет получить решение, соответствующее требованиям
    заказчика, то результаты эксперимента уже показываются техническому эксперту со стороны заказчика,
    чтобы он дополнительно убедился, что реализация данной методики не противоречит другим требованиям
    проектирования генпланов.

    7. ~ X мин X сек

    Для построения генплана требуется учесть большой набор входных данных.
    Часть этих входных отражена в левом столбце на слайде.

    Результатом работы системы является генплан площадного объекта.
    Список из чего состоит генплан указан на слайде в правом столбце.

    8. ~ X мин X сек

    После проведенного интервью с пользователями системы был проведен анализ и сформированы,
    как функциональные требования системы.

    9. ~ X мин X сек

    Так и нефункциональные.
    Рассмотрим эти требования детально далее в презентации.

    10. ~ X мин X сек
    Для грамотного проектирования следует выбрать стек технологий. Он представлен на слайде.

    11. ~ X мин X сек

    Бизнес-часть приложения используется для отображения полученных результатов внешним экспертам со стороны заказчика.

    В расчётной части системы используются объекты внутренней модели данных.
    Эти внутренние объекты уже не могут быть интерпретированы внешними экспертами.

    Внутренняя модель данных подвержена сильным изменениям в процессе исследований.
    Новые алгоритмы решения задачи могут давать куда более качественный результат, но только с учетом того, что
    будут использованы дополнительные структуры данных. В то время как внешняя модель данных относительно стабильна
    и представлена объектами целиком и полностью понятными внешним экспертам.

    Целью данной работы является проектирование и реализация только расчётной части системы, поэтому на всех остальных
    диаграммах представленных в работе рассматривается именно эта часть.

    12. ~ X мин X сек
    Для решения поставленной задачи предлагается следующая архитектура расчётного модуля.
    \begin{itemize}
        \item \textit{Computational Server} -- сервер для запуска расчётных задач.
        \item \textit{User Device} -- пользовательское устройство для взаимодействия с расчётным сервером.
        \item \textit{Web Client} -- веб-клиент для получения данных о расчётных задачах.
        \item \textit{Plan Design Algorithms} -- математическая библиотека, содержащая алгоритмы
        для построения генеральных планов.
        \item \textit{Plan Design Model} -- расчётная модель данных.
        \item \textit{Executor} -- сервис, отвечающий за запуск методов математической библиотеки.
        \item \textit{Storage} -- сервис, отвечающий за чтение/запись расчётной модели данных.
        \item \textit{Orchestrator} -- сервис, контролирующий выполнение расчётных задач.
        \item \textit{Plan Design DB} -- база данных (БД).
    \end{itemize}

    Самой ресурсоёмкой частью приложения является запуск математических методов.
    Сервис \textit{Executor}, отвечающий за это, отвязан от использования других компонент системы
    и может работать полностью изолированно.
    Поэтому, в случае необходимости, возможно оперативное развертывание нескольких экземпляров данного сервиса
    на других вычислительных узлах.

    Так как проект является исследовательским, то основными пользователями является немногочисленная команда проекта.
    Поэтому использование дополнительных архитектурных компонент,
    таких как балансировщики нагрузки и системы автоматического развертывания новых экземпляров приложения
    здесь просто не требуется.

    Однако, основной причиной применения микросервисного подхода в данном случае является изолированность задач с целью
    упрощения проведения исследований, а не с целью упрощения развертывания.

    13. ~ X мин X сек

    На данном слайде представлена диаграмма компонентов сервиса запуска математических методов.
    Наибольшую важность здесь представляет компонент Execution. Именно он отвечает за запуск математических методов
    в отдельном процессе, что удовлетворяет второму нефункциональному требованию.

    14. ~ X мин X сек
    На данном слайде подробно изображен компонент запуска математических методов в отдельном процессе.

    16. ~ X мин X сек
    На данном слайде подробно изображено хранилище расчётных данных.
    Этот сервис обеспечивает выполнение функциональных требований, связанных с загрузкой и хранением данных.

    17. ~ X мин X сек
    На диаграмме изображена архитектура сервиса запуска расчётных задач.
    Именно данный сервис и обеспечивает запуск
    Самым важным модулем, который



\end{document}