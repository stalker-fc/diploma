% Рисунок для титульного слайда.
\titlegraphic{
    \includegraphics[scale=.8]{itmo/logo}
}

% Поля title, author, subject, keywords используются при формировании pdf документа.
% Поэтому их нужно заполнять, даже если вы формируете титульный слайд руками.

% Формат: \title[Короткое название]{Полное название}
\title[ВКР]{
	Проектирование и реализация расчётного модуля для системы автоматического формирования
	генеральных планов площадных объектов капитального строительства
}

%\subtitle[short subtitle]{long subtitle}

% В квадратных скобках короткая запись авторов.
\author[Author]{Степанов Сергей}

%\institute[short institute name]{long institute name}

\where{Санкт-Петербург}
\date{\today}

% Тематика и ключевые слова.
\subject{Генеральный план}
\keywords{Генеральный план площадного объекта}


\begin{frame}[plain]
	\itmopolygons{
	\vfill
		\includegraphics[height=0.2\paperheight]{itmo/logo}
	\vfill
		{\Large \textbf \inserttitle}
	\vfill
		{\textbf \insertauthor}
	\vfill
		Научный руководитель: Пантенков С.А. \par
		Рецензент: Ашихмин И.А.
	\vfill
		{\small \insertplace, \insertdate}
}
\end{frame}
