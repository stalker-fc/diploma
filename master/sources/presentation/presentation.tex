% --------------------------------------------------------------------------
% Шаблон презентации в стилистике Университета ИТМО
% Версия шаблона 2.1. Также шаблон доступен на сайтах:
% https://www.overleaf.com/read/rpkkfchcnbsc
% https://www.overleaf.com/latex/templates/itmo-beamer-theme/fpttrgnmqwsb
% https://github.com/AlexZabashta/ITMO-Beamer-theme
% --------------------------------------------------------------------------

% Внимание!!!
% Этот документ создан для примера использования beamer стиливика.
% Не стоит воспринимать его как урок Latex или beamer!
% Ознакомьтесь с возможностями Latex и beamer (хотя бы базовыми) отдельно.

\documentclass[aspectratio=169]{beamer}
\usepackage{presentation_style}
\usepackage{appendixnumberbeamer}

% Без этой команды он иногда ругается.
\hypersetup{unicode=true}

% Пакет для "русификации" Latex.
\usepackage[english,russian]{babel}

% Чтобы адекватно работало копирование текста из полученной .pdf-ки.
\usepackage{cmap}


% Это нужно, чтобы он называл рисунки без сокращения "рис.".
% Таблицы он называет без сокращения по умолчанию.
\addto\captionsrussian{\renewcommand{\figurename}{Рисунок}}

% Пакет для использования запятой в качестве десятичного разделителя.
% Следите, чтобы в формулах запятые стояли с пробелами, там где они запятые. Например $v = (x, y, z)$
\usepackage{icomma}
\usepackage{graphicx}

% Это единственный пакет для библиографии, который у меня заработал с \footcite шаблоном.
% В презентациях лучше делать её руками через \footnote!
% \usepackage[style=mla]{biblatex}
% \addbibresource{references.bib}



% По умолчанию внизу каждого слайда пишется название презентации (\inserttitle).
% Этот текст можно заменить на другой, например:
\setfootlinetext{\insertsection}

\begin{document}

\thispagestyle{empty}
\vskip 15 mm
\centerline{\footnotesize{\bf{Министерство науки высшего образования Российской Федерации}}}
\centerline{\footnotesize{{ФЕДЕРАЛЬНОЕ ГОСУДАРСТВЕННОЕ АВТОНОМНОЕ ОБРАЗОВАТЕЛЬНОЕ УЧРЕЖДЕНИЕ}}}
\centerline{\small{{ВЫСШЕГО ОБРАЗОВАНИЯ}}}
\centerline{{\bf{«НАЦИОНАЛЬНЫЙ ИССЛЕДОВАТЕЛЬСКИЙ УНИВЕРСИТЕТ ИТМО»}}}
\centerline{{\bf{(Университет ИТМО)}}}
\centerline{Факультет информационных технологий и программирования}

\vskip 30 mm
\centerline{\LARGE{ОТЧЕТ}}
\centerline{\LARGE{О НАУЧНО-ИССЛЕДОВАТЕЛЬСКОЙ РАБОТЕ}}
\vskip 2 mm
\centerline{\large{по теме:}}
\vskip 2 mm
\centerline{\large\bf{ПРОЕКТИРОВАНИЕ И РЕАЛИЗАЦИЯ РАСЧЕТНОГО МОДУЛЯ}}
\vskip 1 mm
\centerline{\large\bf{ДЛЯ СИСТЕМЫ АВТОМАТИЧЕСКОГО ФОРМИРОВАНИЯ}}
\vskip 1 mm
\centerline{\large\bf{ГЕНЕРАЛЬНЫХ ПЛАНОВ}}
\vskip 1 mm
\centerline{\large\bf{ПЛОЩАДНЫХ ОБЪЕКТОВ КАПИТАЛЬНОГО СТРОИТЕЛЬСТВА}}
\vskip 35 mm
\centerline{\large{СПИСОК ИСПОЛНИТЕЛЕЙ}}
\vskip 2 mm
\large{
\noindent
Научный руководитель, \\
Университет ИТМО, \\
факультет информационных технологий и программирования, \\
преподаватель \hskip 112 mm Пантенков С.~А.\\
\vskip 2 mm \noindent
Студент, \\
Направление подготовки 09.04.02 \\
Информационные системы и технологии, \\
Академическая группа M42051 \hskip 80 mm Степанов С.~В.\\
\vfill \hfil \break
\centerline{\large Санкт-Петербург } \centerline{ 2022 }}
\newpage


\subsection*{\large{Выбор архитектуры системы}}
\addcontentsline{toc}{subsection}{Выбор архитектуры системы}

Общий вид системы для автоматического проектирования генеральных планов площадных объектов можно представить в виде
двух независимых друг от друга частей: бизнес и расчётной.
Эти части системы схематично изображены на диаграмме ниже(см. рис\ \ref{pic:architecture__system-diagram}).

\begin{figure}[H]
	\hspace*{-2.5 cm}\includegraphics[width=0.6\textwidth, left]{architecture/pictures/system}
	\caption{Общий вид системы}
	\label{pic:architecture__system-diagram}
\end{figure}
\vskip 5 mm

Бизнес часть приложения используется для отображения полученных результатов внешним экспертам со стороны заказчика.
Именно в этой части определены все правила и те объекты, которые используются в экспертной оценке.
В расчётной части системы содержится большое количество объектов внутренней модели данных, используемых для повышения
качества получаемого решения. Эти внутренние объекты никак не могут быть интерпретированы внешними экспертами.

Также стоит отметить, что внутренняя модель данных подвержена сильным изменениям в процессе исследований.
Новые алгоритмы решения задачи могут давать куда более качественный результат, но только с учетом того, что
будут использованы дополнительные структуры данных. В то время как внешняя модель данных относительно стабильна
и представлена объектами целиком и полностью понятными внешним экспертам.

Целью данной работы является проектирование и реализация только расчетной части системы, поэтому на всех остальных
диаграммах представленных в работе рассматривается именно эта часть.
Для решения поставленной задачи предлагается следующая архитектура расчетного модуля.

\begin{figure}[H]
	\hspace*{-2.5 cm}\includegraphics[width=\textwidth, left]{architecture/pictures/deployment_diagram}
	\caption{Диаграмма размещения}
	\label{pic:architecture__deployment-diagram}
\end{figure}
\vskip 5 mm

\noindent Компоненты представленные на диаграмме размещения:
\begin{itemize}
	\item \textbf{Computational Server} -- сервер для запуска расчётных задач.
	\item \textbf{User Client} -- пользовательское устройство для взаимодействия с расчётным сервером.
	\item \textbf{Web Client} -- веб-клиент для получения данных о расчётных задачах.
	\item \textbf{Orchestrator}
	\begin{itemize}
		\item API -- компонент, предоставляющий REST-API для взаимодействия с сервисом.
		\item Execution -- компонент, контролирующий расчет задачи. Последовательно вызывает этапы выполнения задачи.
	\end{itemize}
	\item \textbf{Storage} -- сервис, отвечающий за чтение/хранение данных расчётных задач.
	\begin{itemize}
		\item API -- компонент, предоставляющий REST-API для взаимодействия с сервисом.
	\end{itemize}
	\item \textbf{Executor} -- сервис, отвечающий за запуск математических методов.
	\begin{itemize}
		\item API -- компонент, предоставляющий REST-API для взаимодействия с сервисом.
		\item Execution -- компонент, осуществляющий запуск метода в отдельном процессе, обработку и сохранение решения.
		\item Math Modules -- математическая библиотека, содержащая методы для расчета генеральных планов.
	\end{itemize}
	\item \textbf{PostgreSQL} -- база данных (БД), используемая для хранения данных для расчётных задач.
\end{itemize}

Таким образом, исходя из данной диаграммы можно выделить пять крупных блоков, требующих детализации.
\begin{enumerate}
	\item Математическая библиотека, содержащая методы для расчета генеральных планов.
	\item Модель расчётных данных, служащая для обмена между сервисами.
	\item Сервис \textbf{Executor}, отвечающий за запуск математических методов.
	\item Сервис \textbf{Storage}, отвечающий за чтение/хранение расчётных данных.
	\item Сервис \textbf{Orchestrator}, отвечающий за запуск расчётных задач.
\end{enumerate}

Рассмотрим перечисленные объекты ниже.

\section{Анализ предметной области}

\begin{frame}
\frametitle{Варианты использования}
\begin{figure}
    \includegraphics[scale=.48]{pictures/usecase}
    \caption{Диаграмма вариантов использования}
\end{figure}
\end{frame}

\begin{frame}
\frametitle{Бизнес-процессы}
\begin{figure}
    \includegraphics[scale=.48]{pictures/common_epc}
\end{figure}
\end{frame}

\begin{frame}
\frametitle{Бизнес-процессы}
\begin{figure}
    \includegraphics[scale=.49]{pictures/analytics_epc}
\end{figure}
\end{frame}


\begin{frame}
\frametitle{Бизнес-процессы}
\begin{figure}
    \includegraphics[scale=.49]{pictures/analytics_epc}
\end{figure}
\end{frame}


\section{Формирование требований}

\begin{frame}
\frametitle{Функциональные требования}
\begin{itemize}
    \item {
        Возможность расчёта генерального плана площадного объекта в автоматическом режиме.
    }
    \item {
        Расчёт генплана должен представлять последовательность этапов.
    }
    \item {
        Результат каждого этапа расчёта должен быть сохранён в долговременное хранилище.
    }
    \item {
        Возможность продолжить расчёт с последнего успешно завершенного этапа.
    }
    \item {
        Возможность сравнения одинаковых расчётных объектов, полученных путем применения различных методик.
    }
    \item {
        Возможность загрузки данных, полученных от технических экспертов, в расчётный модуль.
    }
    \item {
        Возможность загрузки результатов экспериментов, а также информации об особенностях
        проведения экспериментов в расчётный модуль.
    }
\end{itemize}
\end{frame}


\begin{frame}
\frametitle{Нефункциональные требования}
\begin{itemize}
    \item {
        Проведение исследований на вычислительном сервере с операционной системой Ubuntu 20.04 LTS.
    }
    \item {
        Осуществление вызова алгоритмически сложной части системы в отдельном процессе.
    }
    \item {
        Разработанные алгоритмы должны быть оформлены в отдельную библиотеку, имеющей версионирование.
    }
    \item {
        Обеспечение высокой скорости добавления алгоритмических методик в проект.
    }
    \item {
        Обеспечение высокого уровня гибкости системы.
    }

\end{itemize}
\end{frame}

%\subsection*{\large{Выбор архитектуры системы}}
\addcontentsline{toc}{subsection}{Выбор архитектуры системы}

Для решения задачи автоматической генерации генерального плана сооружения предлагается следующая архитектура системы.

\begin{figure}[H]
	\hspace*{-2.5 cm}\includegraphics[width=1.2\textwidth, left]{images/architecture/1}
	\caption{Диаграмма размещения}
	\label{pic:architecture__deployment-diagram}
\end{figure}
\vskip 5 mm

\noindent Компоненты представленные на диаграмме:
\begin{itemize}
	\item \textbf{Web Interface} -- веб-интерфейс, позволяющий осуществлять взаимодействие с системой пользователю.
	\item \textbf{Tile Server Container}
	\begin{itemize}
		\item API -- веб-сервер, предоставляющий API для получения подложки карты, а также векторные данные для
		повышения производительности отображения данных на карте.
%		https://wiki.openstreetmap.org/wiki/Vector_tiles
	\end{itemize}
	\item \textbf{Application Container}
	\begin{itemize}
		\item API -- веб-сервер, предоставляющий GraphQL-API для взаимодействия с системой.
		Также отвечает за предоставление и обработку данных, обеспечивающих выполнение расчета.
	\end{itemize}
	\item \textbf{Database} -- база данных (БД), используемая для хранения данных расчетных задач.
	\item {\bf Task Execution Container}
	\begin{itemize}
		\item Task Manager -- компонент, контролирующий расчет задачи. Последовательно вызывает этапы выполнения задачи.
	\end{itemize}
	\item {\bf Calculation Execution Container}
	\begin{itemize}
		\item Calculation Manager -- компонент, осуществляющий получение и преобразование данных из БД к данным,
		обрабатываемых Calculation Worker-ами, а также за их запуск.
		\item Calculation Worker -- компонент, осуществляющий запуск метода в отдельном процессе, обработку и сохранение решения.
		\item Math Modules -- модули, представляющие собой математическую библиотеку, которая используется Calculation Worker-ом.
		\item Storage -- компонент, осуществляющий чтение/запись входных данных для метода.
	\end{itemize}
\end{itemize}


Расчет генплана площадного объекта -- это комплексная задача, которая состоит из множества различных,
связанных между собой этапов. Результатом задачи является построенный генеральный план.
Генеральный план состоит из следующих объектов:
\begin{itemize}
	\item Местоположение сооружений.
	\item Коммуникационные сети (линии электропередач, трубопроводы).
	\item Эстакады для прокладки коммуникационных сетей.
	\item Внутриплощадочные проезды.
	\item Местоположение площадного объекта на местности.
	\item Зоны теплового излучения.
	\item Зоны поражения взрывных волн.
	\item Объёмы выемки/отсыпки строительной площадки.
	\item Ограждения групп сооружений.
\end{itemize}

Вычисление всех этих объектов представляет последовательный вызов расчетных методов.
Каждый из этих методов выполняет свою определенную задачу: один рассчитывает местоположения зданий, другой, на основе
местоположений зданий рассчитывает конфигурацию внутриплощадочных проездов. Каждый из методов влияет на
результаты других методов. Поэтому разная последовательность их вызовов может давать кардинально разные результаты.

Расчет каждого метода может продолжаться длительное время (например: больше одного часа). Поэтому очень важно
чтобы результаты расчетов методов для задачи были сохранены. В случае возникновения ошибки в процессе расчета задачи,
нельзя допустить, чтобы результаты успешно пройденных этапов были потеряны. Требуется, чтобы после исправления ошибки
в расчете можно было продолжить выполнение задачи.

Поэтому расчет задачи выполняется в двух отдельных контейнерах: \textit{Task Execution Container} отвечает за оркестрацию
вызываемых методов для расчета, а \textit{Calculation Execution Container} за запуск методов из математической библиотеки
в отдельных процессах.

Для того чтобы снизить временные издержки при разработке, будем упрощать интерфейсы взаимодействия между компонентами.
Так как все данные хранятся в базе данных, то нам потребуется реализовать интерфейс взаимодействия между БД
и программным кодом. Данный интерфейс можно будет переиспользовать в трех контейнерах, что позволит сэкономить время на
разработку.

Так как каждый контейнер -- это отдельный процесс, то при запуске задач мы решаем проблему межпроцессного взаимодействия.
Самый простой интерфейс сетевого взаимодействия между двумя процессами -- это использование таблицы в базе данных.
Мы можем применить данный способ потому что у нас малое количество рассчитываемых задач, поэтому проблемы
с использованием БД, как брокера сообщений, просто не возникает.


%\section*{\Large{Реализация}}
\addcontentsline{toc}{section}{Реализация}

При реализации описанной архитектуры была получена следующая структура проекта.
\dirtree{%
.1 /.
.2 app.
.3 api.
.4 \_\_init\_\_.py.
.4 contexts.py.
.4 controller.py.
.4 model.py.
.4 responses.py.
.4 views.py.
.3 domain.
.4 \_\_init\_\_.py.
.4 cache\_storage.py.
.4 data\_provider.py.
.4 model.py.
.4 renderer.py.
.4 repository.py.
.3 execution.
.4 \_\_init\_\_.py.
.4 executor.py.
.4 handler.py.
.4 listener.py.
.4 queue.py.
.4 result.py.
.4 storage.py.
.3 repository.
.4 \_\_init\_\_.py.
.4 cache\_storage.py.
.4 data\_provider.py.
.4 renderer.py.
.4 repository.py.
.3 \_\_init\_\_.py.
.3 app.py.
.3 exceptions.py.
.3 logger.py.
.2 Dockerfile.
.2 README.md.
.2 main.py.
.2 requirements.txt.
}

Само приложение находится в директории \textbf{app}.
Всего получилось 4 python-пакета, в которых и скрыта основная логика работы.
\begin{enumerate}
    \item \textbf{api} -- реализует API тайлового сервера через HTTP. То есть endpoint-ы,
    модели запросов/ответов к серверу, а также взаимодействие с \textit{ITilesRepository}
    \item \textbf{domain} -- пакет, в котором определены только интерфейсы взаимодействия между модулями программы.
    Именно здесь и реализована диаграмма классов (см. рис\ \ref{pic:architecture__classes-diagram}). Непосредственная
    реализация обозначенных интерфейсов находится в других пакетах.
    \item \textbf{execution} -- пакет, в котором вызывается код, отвечающий за рендеринг в отдельном процессе.
    Рендеринг является CPU-bound задачей, поэтому правильнее
    запускать его через отдельный процесс, чтобы он не воздействовал на главный процесс приложения.
    \item \textbf{repository} -- в этом пакете находится реализация интерфейсов из \textbf{domain}.
\end{enumerate}

\section*{\Large{Примеры кода}}
\addcontentsline{toc}{section}{Примеры кода}

\begin{lstlisting}[language=Python, caption=main.py, captionpos=b]
from app import startup_app

if __name__ == '__main__':
    from aiohttp import web

    web.run_app(startup_app(), host="0.0.0.0", port=8080)
\end{lstlisting}
\vskip 10 mm
\begin{lstlisting}[caption=Dockerfile, captionpos=b]]
FROM python:3.8@sha256:4c4e6735f46e7727965d1523015874ab08f71377b3536b8789ee5742fc737059

WORKDIR /app

ENV LC_ALL C.UTF-8
ENV LANG C.UTF-8
ENV N_WORKERS 8

COPY requirements.txt .
RUN pip3 install --no-cache-dir -r requirements.txt

RUN pip3 check

COPY main.py .

COPY /app ./app

ENTRYPOINT /bin/bash -c "gunicorn run_app:app --workers=${N_WORKERS} --bind 0.0.0.0:8080 --worker-class aiohttp.GunicornWebWorker --timeout 0"
\end{lstlisting}
\vskip 10 mm
\begin{lstlisting}[language=Python, caption=domain/model.py, captionpos=b]
from enum import Enum
from typing import Dict
from typing import List
from typing import Type
from typing import Union
from uuid import UUID

from dataclasses import dataclass
from shapely.geometry.base import BaseGeometry


@dataclass
class TileInfo:
    site_plan_id: UUID
    x: int
    y: int
    z: int

class GeometryType(Enum):
    POINT = "point"
    LINESTRING = "line"
    POLYGON = "polygon"

@dataclass
class Feature:
    geometry: Type[BaseGeometry]
    properties: Dict[str, Union[int, float, str, bool]]

@dataclass
class Layer:
    name: str
    geometry_type: GeometryType
    features: List[Feature]

@dataclass
class TileData:
    layers: List[Layer]


@dataclass
class LayerConfiguration:
    layer_name: str
    geometry_type: GeometryType
    sql_query: str

@dataclass
class TileConfiguration:
    layers_configurations: List[LayerConfiguration]
\end{lstlisting}

\vskip 10 mm

\begin{lstlisting}[language=Python, caption=domain/model.py, captionpos=b]
import traceback
from pathlib import Path
from uuid import UUID

from aiohttp import web

from app.exceptions import TileDoesNotExistException
from app.domain.model import TileInfo
from app.repository import create_tiles_repository


async def index(request):
    static_dir = request.config_dict['static_dir']
    return web.FileResponse(static_dir / Path('index.html'))


async def vector_tiles(request):
    tile = TileInfo(
        site_plan_id=UUID(request.rel_url.query['task_id']),
        x=int(request.rel_url.query['x']),
        y=int(request.rel_url.query['y']),
        z=int(request.rel_url.query['z'])
    )

    vector_data_service = request.config_dict['vector_data_service']
    try:
        tile_data: bytes = tiles_repository.get_tile(tile)
        return web.Response(
            body=tile_data,
            status=200,
            content_type="application/x-protobuf",
            headers={
                'Access-Control-Allow-Origin': '*',
            }
        )
    except TileDoesNotExistException as e:
        return web.Response(
            status=404,
            text=str(e),
            headers={
                'Access-Control-Allow-Origin': '*'
            }
        )
    except Exception as e:
        tb = traceback.format_exc()
        print(tb)
        return web.Response(
            status=500,
            text=tb,
            headers={
                'Access-Control-Allow-Origin': '*'
            }
        )


async def clean_cache_for_task(request):
    task_id = UUID(request.rel_url.query['task_id'])

    tiles_repository = request.config_dict['tiles_repository']
    tiles_repository.clean(task_id)
    return web.Response(
        status=200,
        headers={
            'Access-Control-Allow-Origin': '*',
        }
    )


def startup_app():
    app = web.Application()

    tiles_repository = create_tiles_repository()
    static_dir = Path(__file__).parent / Path('static')

    app['tiles_repository'] = tiles_repository
    app['static_dir'] = static_dir

    app.add_routes([
        web.get("/", index),
        web.get("/index", index),
        web.get("/vector", vector_tiles),
        web.get("/clean", clean_cache_for_task),
    ])
    return app
\end{lstlisting}
%
%\section{Результаты}

\begin{frame}
\frametitle{Результаты}
\begin{itemize}
    \item собраны и проанализированы требования пользователей;
    \item сформированы функциональные и нефункциональные требования к программному компоненту;
    \item cпроектирована системная и программная архитектура расчётного модуля;
    \item {
           реализован расчётный модуль, состоящий из пяти программных компонент:
            \begin{enumerate}
                \item математическая библиотека,
                \item расчётная модель данных,
                \item сервис запуска расчётных задач,
                \item сервис хранения расчётных данных,
                \item сервис запуска математических методов.
            \end{enumerate}
    }
\end{itemize}
\end{frame}


%
%\begin{frame}[plain]
    \itmopolygons{
        \vfill
        \Huge{Спасибо за внимание!}
        \vfill
        \includegraphics[scale=.5]{itmo/slogan}
    }
\end{frame}

\appendix
\section{Приложения}

\begin{frame}
\frametitle{Тестирование}
\begin{figure}
Общее покрытие тестами составляет около 50\%.
\begin{itemize}
    \item Математическая библиотека 53\%
    \item Сервис запуска расчётных задач 46\%
    \item Сервис запуска математических методов 54\%
    \item Сервис хранилища расчётных данных 49\%
\end{itemize}
\end{figure}
\end{frame}

\begin{frame}
\frametitle{Варианты использования}
\begin{figure}
    \includegraphics[scale=.48]{pictures/analysis/usecase}
    \caption{Диаграмма вариантов использования}
\end{figure}
\end{frame}


\begin{frame}
\frametitle{Предметная область задачи}
\begin{columns}[c]

\column{.45\textwidth}{
    Входные данные
    \begin{itemize}
        \item допустимая для строительства область на карте;
        \item стоимостная модель расчета стоимости инженерной подготовки;
        \item перечень сооружений;
        \item параметры коммуникаций между сооружениями проектируемого объекта;
        \item параметры цифровой модели рельефа;
    \end{itemize}
}

\column{.45\textwidth}{
    Выходные данные
    \begin{itemize}
        \item фигура площадного объекта;
        \item местоположения сооружений;
        \item схема технологических эстакад минимальной длины;
        \item схема внутриплощадочных проездов;
        \item стоимость инженерной подготовки;
        \item зоны распространения теплового потока;
        \item зоны распространения взрывной волны;
    \end{itemize}
}
\end{columns}
\end{frame}


\begin{frame}
\frametitle{Технологический стек}
\begin{itemize}
    \item Операционная система \textit{Ubuntu 20.04 LTS}
    \item Язык программирования \textit{Python 3.8.12}
    \item База данных \textit{PostgreSQL 12} с расширением \textit{PostGIS 3.1}
    \item Веб-сервер \textit{nginx}
    \item Автоматическое развертывание \textit{Gitlab-CI}
    \item Контейнеризация \textit{Docker} и \textit{docker-compose}
    \item Документация \textit{OpenAPI} и \textit{LaTex}
    \item Система сборки логов \textit{ELK}
    \item Протоколы взаимодействия \textit{REST API over HTTP}
    \item Формат данных \textit{JSON}
\end{itemize}
\end{frame}

\begin{frame}
\frametitle{Диаграмма развёртывания}
\begin{figure}
    \includegraphics[scale=.49]{pictures/architecture/deployment}
\end{figure}
\end{frame}

\begin{frame}
\frametitle{Сервис запуска математических методов}
\begin{figure}
    \includegraphics[scale=.6]{pictures/architecture/executor_component_common}
\end{figure}
\end{frame}

\begin{frame}
\frametitle{Компонент запуска математических методов}
\begin{figure}
    \includegraphics[scale=.5]{pictures/architecture/executor_component_detailed}
\end{figure}
\end{frame}

\begin{frame}
\frametitle{Хранилище расчётных данных}
\begin{figure}
    \includegraphics[scale=.55]{pictures/architecture/storage_component_common}
\end{figure}
\end{frame}


\begin{frame}
\frametitle{Компонент запуска расчётных задач}
\begin{figure}
    \includegraphics[scale=.5]{pictures/architecture/orchestrator_component_detailed}
\end{figure}
\end{frame}


%\input{samples.tex}

\end{document}
