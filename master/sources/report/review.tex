\subsection*{\Large{Реферат}}
5.3.1 Общие требования к реферату на отчет о НИР - по ГОСТ 7.9.

5.3.2 Реферат должен содержать:

- сведения об объеме отчета, количестве иллюстраций, таблиц, приложений, количестве частей отчета, количестве использованных источников;

- перечень ключевых слов;

- текст реферата.

5.3.2.1 Перечень ключевых слов должен включать от 5 до 15 слов или словосочетаний из текста отчета, которые в наибольшей мере характеризуют его содержание и обеспечивают возможность информационного поиска. Ключевые слова приводятся в именительном падеже и печатаются прописными буквами в строку через запятые.

5.3.2.2 Текст реферата должен отражать:

- объект исследования или разработки;

- цель работы;

- метод или методологию проведения работы;

- результаты работы и их новизну;

- основные конструктивные, технологические и технико-эксплуатационные характеристики;

- степень внедрения;

- рекомендации по внедрению или итоги внедрения результатов НИР;

- область применения;

- экономическую эффективность или значимость работы;

- прогнозные предположения о развитии объекта исследования.

Если отчет не содержит сведений по какой-либо из перечисленных структурных частей реферата, то в тексте реферата она опускается, при этом последовательность изложения сохраняется.

5.3.2.1, 5.3.2.2 (Измененная редакция, Изм. N 1).

5.3.3 Пример составления реферата приведен в приложении А.
