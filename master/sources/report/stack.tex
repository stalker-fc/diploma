\subsection*{\large{Выбор стека технологий}}
\addcontentsline{toc}{section}{Выбор стека технологий}
Во время проектирования \textit{системы в целом} всегда возникает ряд вопросов,
связанных с выбором стека технологий для решения следующих задач.

\begin{enumerate}
    \item Реализация программного кода
    \item Хранение данных
    \item Развертывание
    \item Тестирование
    \item Логирование
    \item Документация
    \item Протоколы взаимодействия
\end{enumerate}

Рассмотрим каждую из задач более подробно и выберем технологию, которая наиболее подходит для её решения в нашем случае.

\noindent \textbf{1. Язык программирования}.

В качестве основного языка программирования был выбран \textit{Python 3.8}.
% https://docs.python.org/3.8/
Самым ценным ресурсом при реализации нашей системы является время разработки. Python является языком программирования
с динамической типизацией, имеет простой синтаксис,
а также очень большую базу готовых библиотек для решения любых задач.
Все эти преимущества позволяют, как быстро реализовывать функционал со стороны разработки, так и проводить проверку
разных алгоритмических гипотез за короткое время.

Такие недостатки Python, как низкая скорость выполнения программ и большое потребление памяти, в нашем случае
не играют решающей роли.
% https://techvidvan.com/tutorials/python-advantages-and-disadvantages/
В Ubuntu 20.04 LTS -- операционной системе, на которой ведется разработка, установлен по умолчанию Python 3.8, поэтому
была выбрана именно эта версия языка.

\noindent \textbf{2. Хранение данных}.

Все данные, которые используются в проекте, достаточно однородны и могут быть легко представлены в табличном виде.
Поэтому в качестве технологии для хранения данных будем использовать реляционную базу данных.
Так как проектирование генеральных планов напрямую связано с геоданными, то необходимо учитывать это требование.

В качестве технического решения предлагается связка \textit{PostgreSQL 12} и расширения для работы с геоданными
\textit{PostGIS 3.1}. PostgreSQL является популярным open-source решением в мире баз данных,
а PostGIS содержит очень большое количество функций для преобразований геоданных, что нам очень важно.
% https://postgis.net/documentation/
В версии PostGIS 3.0 появилась поддержка Mapbox Vector Tile, что позволяет генерировать данные для визуализации с помощью
стандартных функций базы данных.

\noindent \textbf{3. Развертывание}.

Требуется

Так как внутри компании используется

\noindent \textbf{4. Тестирование}.
Виды тестирования:
- Модульное тестирование. Используется библиотека pytest для языка Python 3.
- Функциональное тестирование. Используются библиотеки pytest и behave для языка Python 3. Библиотека behave используется для реализации BDD-подхода в тестировании.
- Интеграционное тестирование. Используется библиотека pytest для языка Python 3.

\noindent \textbf{5. Логирование}.
ELK

\noindent \textbf{6. Документация}.
Sphinx, GraphQL

\noindent \textbf{7. Протоколы взаимодействия}.
В целях уменьшения общего времени разработки
