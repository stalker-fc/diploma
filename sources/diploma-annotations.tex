\newpage
Название работы: Оптимизация планов капитального строительства линейных объектов на базе многофакторного анализа с учетом геологических ограничений

В данной работе рассматривается задача построения оптимальной по затратам на строительство сети между набором точек на заданном участке поверхности Земли. При построении сети учитывается рельеф, стоимость строительства на различных типах местности, а также уже существующая инфраструктура. Для решения данной задачи в работе используется граф, рассматриваются различные способы его генерации, а также предлагается алгоритм по построению оптимальных сетей, на основе знаний о предметной области.

Ключевые слова: графы, задача Штейнера, кратчайший путь, расчётная сетка, оптимизация

\newpage
Diploma: Optimization linear objects alignment for infrastructure design considering variety of factors and geological constraints

Abstract: In this paper, we consider the problem of a network infrastructure design between a set of points which are located on the Earth's surface. We take into account the Earth`s relief, different types of terrain and existing infrastructure. We consider on variety of mesh generation ways to create the graph to solve this problem. Also we propose an algorithm of design networks based on the knowledge of the subject area.

Keywords: graphs, Steiner problem, shortest path, mesh generation, optimization
