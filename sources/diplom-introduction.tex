\newpage
\section*{\LARGE{Введение}}
\addcontentsline{toc}{section}{Введение}
\hskip 12 mm
В современном мире каждая компания хочет минимизировать свои затраты, как по времени, так и в финансовом плане. Это не обошло и сферу строительства. В данной области существует ряд открытых задач, требующих решения, в том числе это касается построения оптимальных траекторий линейных объектов.
\par
Объектами капитального строительства в данной работе называются площадные объекты, такие как здания, строения, сооружения, и линейные объекты, такие как линии электропередачи, линии связи, трубопроводы, автомобильные дороги, железнодорожные линии и другие подобные сооружения \cite{CapitalBuilding}.
\par
Во время строительства линейного объекта человеку приходится сталкиваться с преодолением естественных препятствий, таких как леса, болота, реки, а также препятствий, представляющих собой разного рода сооружения и области, на которых строительство запрещено по юридическим и иным причинам. Эти ограничения на строительство называются инженерно-геологическими условиями и включают в себе несколько составляющих, например: 
\begin{itemize}
	\item геологическое строение местности;
	\item рельеф;
	\item гидрогеологические условия;
	\item мерзлотные условия;
	\item современные геологические процессы \cite{EngGeoConditions}.
\end{itemize}
\par
Помимо нахождения оптимальной траектории линейного объекта, соединяющего две точки, возникает проблема проектирования набора оптимальных трасс линейных объектов для сети, соединяющих большее количество точек. Кроме построения оптимальной сети линейных объектов, учитывая только затраты на строительство, имеет смысл проектировать сеть линейных объектов, опираясь также на затраты по её эксплуатации и ряд других факторов, так как именно поддержание инфраструктуры в работоспособном состоянии может отнимать наибольшее количество средств.