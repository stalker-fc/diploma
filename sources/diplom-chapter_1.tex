\newpage
\section*{\LARGE{Глава 1. Метрики}}
\addcontentsline{toc}{section}{Глава 1. Метрики}
\hskip 12 mm
В связи с тем, что мы работаем над созданием системы, которая будет использоваться в реальном мире, то нужно понять, каким образом можно представлять данные, используемые для решения задачи. Так как в нашем продукте при строительстве линейного объекта должен учитываться рельеф, то каждую точку земной поверхности можно представить в виде:\\
\centerline{ $p =\{lat, long, height\}$,}\\ 
где $lat$ - широта точки, $long$ - долгота точки, $height$ - геодезическая высота.\\
Для получения высоты точки на поверхности Земли используются цифровые модели рельефа.
\subsection*{\Large{1.1 Цифровая модель рельефа}}
\addcontentsline{toc}{subsection}{1.1 Цифровая модель рельефа}
\hskip 12 mm
Цифровая модель рельефа --- это математическое представление участка земной поверхности, полученное путём обработки материалов топографической съёмки. Представляет собой информацию о высоте поверхности земли без учёта растительности, зданий и других высотных объектов, которые на ней находятся \cite{DEM}.
\par
Существуют разные источники данных цифровой модели местности. У каждого из них есть свои преимущества и недостатки (см. табл. \ref{tabular:dem}).  
\begin{table}[H]
	\centering
	\caption{Примеры источников данных цифровой модели местности}
	\label{tabular:dem} 
	\begin{tabular}{|c|c|p{8cm}|c|}
		\hline
		\textbf{Источник}     &  \textbf{Разрешение} & \textbf{Ограничение} & \textbf{Стоимость} \\ \hline
		SRTM \cite{SRTM} & 90 м & Имеются данные высот только между 60° с.ш. и 54° ю.ш. & Бесплатно  \\ \hline
		ASTER GDEM \cite{ASTER} & 15 м & Имеются достаточно большие шумы, ухудшающие возможность практического использования. Данные высот между 83° с.ш. и 83° ю.ш.  & Бесплатно  \\ \hline
		GMTED2010 \cite{GMTED2010} & 250 м & Имеются данные высот для всей поверхности планеты, но в достаточно низком разрешении.  & Бесплатно  \\ \hline
		WorldDEM \cite{WorldDEM} & 12 м & Без ограничений. Есть данные высот для всей территории планеты. & Платно  \\ \hline
	\end{tabular}
\end{table}
\vspace{4mm}
\par
В данной работе будем использовать данные SRTM, так как исследуемые территории попадают в указанные ограничения по широтам, и точности данной цифровой модели рельефа достаточно для наших целей. Помимо этого  для языка программирования Python существует удобный модуль SRTM.py, служащий для получения высоты точки, заданной географическими кооординатами. В остальных случаях требуется скачивание и дополнительная подготовка данных цифровых моделей рельефа.
\subsection*{\Large{1.2 Измерение расстояний}}
\addcontentsline{toc}{subsection}{1.2 Измерение расстояний}
\hskip 12 mm
Фигура Земли имеет форму геоида. Геоид --- выпуклая замкнутая поверхность, примерно совпадающая с поверхностью воды в морях и океанах в спокойном состоянии и перпендикулярная к направлению силы тяжести в любой её точке \cite{Geoid}.
\par
Земной эллипсоид --- эллипсоид вращения, размеры которого подбираются при условии наилучшего соответствия фигуре квазигеоида для Земли в целом или отдельных её частей \cite{Ellipsoid}. Все координаты и дистанции вычисляются именно на эллипсоиде.
\par
Референц-эллипсоид --- приближение формы поверхности Земли эллипсоидом вращения, используемое для нужд геодезии на некотором участке земной поверхности \cite{RefEllipsoid}.
\par
Нахождение расстояние между двумя точками --- это решение обратной геодезической задачи. Есть несколько способов измерения расстояния между двумя точками, лежащими на земной поверхности:
\begin{itemize}
	\item Вместо использования географической системы координат, можно использовать метрическую систему координат, например, Universal Transverse Mercator (UTM). Это позволит вычислять расстояние между точками, используя Евклидову метрику. В случае использования данного способа требуется подбирать датум, то есть набор параметров, используемых для смещения и трансформации референц-эллипсоида в локальные географические координаты с целью снижения ошибки в расчетах, для каждой конкретной области, так как при использованиии датума для общеземного эллипсоида WGS84 относительная ошибка измерения может превышать 100\% при измерении расстояний в областях близких к полярным широтам.
	\item Использование метрического тензора $S = \int \sqrt{g_{\alpha\beta}(x) dx_\alpha dx_\beta}$,\\ где $g={\begin{bmatrix}R^{2}&0\\0&R^{2}\sin ^{2}\theta \end{bmatrix}}$. Данный способ хорошо работает только на достаточно близких расстояниях, а вот на дистанции около 10 км и больше данный метод дает погрешность более 2\%.
	\item Формула гаверсинусов \cite{Haversine}. При использовании данной формулы мы пользуемся моделью Земли в виде сферы и с помощью тригонометрических преобразований находим кратчайшее расстояние между точками. Данный способ дает погрешность порядка 0.3\%.
	\item Формула Винценти \cite{Vincenty}. При использовании данной формулы  мы представляем фигуру Земли в форме эллипсоида и вычисляем расстояние с помощью итерационной формулы. Данная формула является наиболее точной для вычисления расстояния и дает погрешность около в 1 мм вне зависимости от удаленности точек.
\end{itemize}
\par
Для выбора оптимального способа, нужно учитывать два параметра: время вычисления и точность измерения. Первые два способа мы не будем использовать в связи с отсутствием универсальности, а для третьего и четвертого способа проведем замеры производительности. Создадим тестовую выборку из десяти тысяч точек со случайно выбранными координатами и проведем измерение расстояний для каждой пары точек двумя разными способами. В среднем время вычисления расстояния третьим способом в 4 раза меньше, чем четвертым, и составляет $4 \cdot 10^{-6}$ сек против $16 \cdot 10^{-6}$ сек для каждой пары точек. Таким образом, оптимальным выбором является вычисление расстояния с помощью формулы гаверсинусов. Но все формулы представленные выше не учитывают рельеф, поэтому при окончательном расчете длины пути потребуется добавить также и перепады высот между двумя точками.
\subsection*{\Large{1.3 Вычисление стоимости пути}}
\hskip 12 mm
\addcontentsline{toc}{subsection}{1.3 Вычисление стоимости пути}
Для того чтобы оптимизировать расположение линейного объекта требуется уметь вычислять его стоимость строительства. Каждый линейный объект можно представить в виде состоящей из отрезков ломаной $L = \{\{A_i, B_i\}_{i=0}^n$, где $A_i, B_i \in Q\}$. Стоимость строительства каждого прямого отрезка $AB_i$ можно представить в виде значения функции $C_{AB_i}$, где каждый отрезок задается параметрически:
\par
${\displaystyle AB_i\colon \left\{
	{\begin{aligned}
			&x=x\left(t\right),\\
		    &h=h\left(x(t)\right),\\
			&c=c\left(x(t)\right)\\
	\end{aligned}}\right.\qquad t\in \left[0, 1\right]}$, где $ \left[0, 1\right]$ - отрезок параметризации.

$C_{AB_i}$ = $\int \limits_{AB_i} c\left(x(t)\right) \sqrt{\left[ g_{\alpha\beta}(x(t)) + \frac{\partial h}{\partial x_\alpha} \cdot \frac{\partial h}{\partial x_\beta}\right] \dot{x}_\alpha \dot{x}_\beta} dt$, \\ где $g$ --- метрический тензор поверхности Земли без учета рельефа.
\par
Таким образом стоимость строительства линейного объекта можно представить в виде суммы 
$C_L = \sum_{i = 0}^{n} C_{AB_i}$, а стоимость строительства сети линейных объектов, как 
$C_N = \sum_{j = 0}^{m} C_{L_j}$. Общее решение задачи сводится к поиску минимальной по стоимости сети среди всех построенных сетей линейных объектов $N_{res} = {\underset  {{N}}{\operatorname {argmin}}}(C_N)$.
\subsection*{\Large{1.4 Вычисление стоимости строительства в точке}}
\addcontentsline{toc}{subsection}{1.4 Вычисление стоимости строительства в точке}
\hskip 12 mm
Так как по условию задачи все области и линейные объекты заданы в виде набора вершин, лежащих на плоскости $Q$, то функцию стоимости строительства в точке можно определить через принадлежность точки определенному многоугольнику. У любого существующего линейного объекта есть ширина, поэтому при начальной инициализации алгоритма следует провести операцию по превращению линейного объекта в многоугольник. 
\par
Зачастую данные, получаемые на вход программой, являются несовершенными и содержат взаимопепересечения, как линейных объектов, так и областей. Поэтому функцию строительства в точке определим, как максимальное значение стоимости, среди всех областей, в которых лежит точка. И если точка лежит в области, принадлежащей линейному объекту, то стоимость строительства в этой точке будет рассчитываться, как минимальная стоимость среди всех линейных объектов, в которых лежит точка. То есть формально:

$\begin{cases}
	min(L_{cost}), & \text{если точка лежит на линейном объекте}; \\
	max(A_{cost}), & \text{иначе}. \\
\end{cases}$

Существует два подхода для вычисления стоимости строительства в каждой точке области $Q$: используя векторные данные и используя растр.
\par
Векторный способ получения стоимости строительства в точке предполагает проверку принадлежности каждой точки всем многоугольникам, которые лежат в области проектирования $Q$. Основными преимуществами данного способа является очень высокая точность определения стоимости относительно координат точки и малый объем занимаемой памяти. Главным минусом является низкое быстродействие, которое можно повысить, используя алгоритмы на основе пространственной сетки \cite{SpatialIndex}.
\par
Вторым способом получения стоимости строительства в точке, является использование растра. Растр --- это матрица, каждое значение которой представляет собой проекцию набора стоимости точек области проектирования $Q$, выбранных определенным способом. Из главных плюсов этого способа хранения --- это быстродействие, ведь для того чтобы получить значение стоимости в точке, нужно лишь вычислить индексы этой точки для матрицы растра. Но минусами этого способа являются низкая точность определения стоимости точки, так как тут возникают абсолютно те же проблемы, как и при представлении линии в растровом изображении на компьютере. А вторым недостатком является очень большой объем занимаемой памяти, так как чтобы нивелировать первую проблему, требуется повысить точность растра, увеличив число точек в нём.


