\Large{\bf{План диплома}}
\begin{itemize}
\item Введение.\\
Общие слова о том, что будет в дипломе, определения, которые указаны в самой теме диплома
\item Постановка задачи\\
Актуальность проблемы. Здесь идет рассказ о том, как обычному инженеру плохо живется, как понять, что выгоднее - обойти болото или пройти напрямик, как вообще разобраться с рельефом местности, также рассказать про геологию, а еще про разные метрики: о том, что кратчайшая дорожная сеть может и не быть самой выгодной для эксплуатации(Задача Вебера)
\item Обзор литературы.\\
 Пишем, что проблема существует достаточно давно, что ей занимался сначала Ферма, потом Вебер, но все это было на сугубо математическом уровне. Эта же тема получила и активное развитие в 20 веке, как со стороны зарубежных ученых, так и со стороны советских. Зарубежные ученые делали больший упор на евклидову метрику, в то время, как Лотарев пытался решить реальную задачу для построения сети месторождения. Эта тема до сих пор интересна научному сообществу, о чем свидетельствует DIMACS 11 - конференция-соревнование проходившее в конце 2014 года, да и работа МФТИ 2017 года показывает актуальность этой проблемы до сих пор.
\item Глава 1. Измерение расстояний\\
1.1. Что представляют собой точки маршрута. Высота точки, разные источники получения данных высот.\\ 1.2. Земля сфера, разные проекции, разный способ вычисления длины.\\
1.3. Разные стоимости областей, интеграл по этим областям - и есть стоимость отрезка\\
*1.4. Растр или вектор для хранения стоимостей. Достоинства и недостатки.
\item Глава 2. Описание методов поиска кратчайшего пути\\
2.1. Здесь рассказываем почему граф такой классный и почему методы оптимизации - это боль.\\
2.2. Способы нахождения кратчайшего пути на графе, почему А* плохо работает в нашем случае.
\item Глава 3. Генерация графа\\
3.1.Определение триангуляции Делоне\\
3.2.Взвешивание ребер графа\\
3.3.Насколько случайное смещение сетки влияет на качество пути. Насколько размер сетки влияет на качество пути. Насколько сглаживания улучшает качество пути.\\
3.4.Описание тестовых кейсов для поиска кратчайшего пути между двумя точками.\\
*3.5. Если будет растр, то можно попробовать подобрать эвристическую функцию
\item Глава 4. Построение оптимальной сети для n-объектов.\\
4.1. Я не знаю, что здесь писать. Можно сделать типа обзор литературы по этой тематике, но ни один из алгоритмов оттуда не был реализован, а был придуман свой\\
4.2. Описание моего алгоритма\\
4.3. Описание тестовых случаев для его тестирования
\item Выводы. Заключение
\item Список литературы
\item Приложения
\end{itemize}
\newpage